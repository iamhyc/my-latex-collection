\documentclass{article}
\usepackage{amsmath}
\usepackage{amssymb}
\usepackage{braket}
\usepackage{xcolor}
\usepackage{enumitem}
\usepackage[normalem]{ulem} % for strikeout line
% \usepackage{graphicx}
% \usepackage{epstopdf}

%-------------------------------------------------------%
\newcounter{pcounter}                                   %
\newenvironment{problem}                                %
{                                                       %
    \color{gray}                                        %
    \stepcounter{pcounter}                              %
    \textbf{\arabic{pcounter}.}                         %
}{}                                                     %
\newenvironment{solution}                               %
{\textbf{Solution:} \\}{$\blacksquare$\newline}         %
%-------------------------------------------------------%
\newcommand{\tab}{\ \ \ \ }                             %
\newcommand{\leadto}{\Rightarrow}                       %
\newcommand{\domR}{\mathbb{R}}                          %
\newcommand{\domS}{\mathbb{S}}                          %
\newcommand{\abss}[1]{\| #1 \|}                         %
\newcommand{\tr}[1]{\textbf{tr}(#1)}                    %
\newcommand{\vecOne}{\textbf{1}}                        %
%-------------------------------------------------------%

\begin{document}
    %------------------- The Title -------------------%
    \parindent 0in
    \parskip 1em
    \title{COMP8802 Assignment 2 Solution Sheet}
    \author{3030058647, HONG Yuncong}
    \maketitle

    %=================== Problem 1 ===================%
    \begin{problem}
        Let S=$acggtcgt$ and T=$acctgtt$. Define V[i,j] as the edit distance between S[1..i] and T[1..j]; s[1..0] (T[1..0]) as empty string; and V[0, 0] = 0. Show all entries in the V table and give the edit distance between S and T.
    \end{problem}

    \begin{solution}
        
    \end{solution}

    %=================== Problem 2 ===================%
    \begin{problem}
        Design and OT${^3}_1$ (oblivious transfer) protocol such that Alice has three messages m$_1$, m$_2$ and m$_3$.
        She wants to let Bob obtain \underline{exactly one} of the messages with requirements that: Alice doest not know which message Bob selects and Bob cannotknow any information about the meesages except the one he selects.
        Show all steps of the protocol (You do NOT need to argue that the protocol satisfies the requirements).
        \\
        {[Bonus: Can you extend it to an OT${^3}_2$ protocol, i.e., Alice wants to let Bob obtain exactly two of the messages with the same security requirements?]}
    \end{problem}

    \begin{solution}
        
    \end{solution}

    %=================== Problem 3 ===================%
    \begin{problem}
        Let S and T be DNA strings (strings with characters \{a, c, g, t\}) of 3 characters.
        Alice holds S and Bob holds T. If we want to use the secure pattern matching protocol discussed in the lecture (Protocol 1, i.e., Alice creates a garbled circuit for the edit distance of S and T and passes it to Bob) to compute the edit distance of S and T;
        Answer the following questiongs (NO NEED to explain your answers).
        \\
        \begin{enumerate}[label=(\alph*)]
            \item How manu input bits are required?
            \item Provide an encoding of the characters \{a, c, g, t\}.
            \item What are the possible values for the edit distance?
            \item How manu output bits are required?
            \item What is the number of rows in the truth table for computing the edit distance of S and T?
            \item How many encryptions Alice has to do?
            \item How many random numbers Alice has to generate?
            \item How many random numbers Alice has to pass to Bob?
            \item How many OT${^2}_1$ oblivious tansfer protocols they have to execute?
            \item How many decryptions Bob has to do?
        \end{enumerate}
    \end{problem}

    \begin{solution}
        
    \end{solution}

    %=================== Problem 4 ===================%
    \begin{problem}
        Provide the whole truth table for the equality circuit discussed in the lecture (Slide 37) for the secure string matching problem. We assume that S and T, each has only 3 characters.
    \end{problem}

    \begin{solution}
        
    \end{solution}

    %=================== Problem 5 ===================%
    \begin{problem}
        Provide the whole truth table for the Min3 circuit discussed in the lecture (Slide 37) for the secure string matching problem. We assume that S and T, each has only 3 characters.
    \end{problem}

    \begin{solution}
        
    \end{solution}

    %=================== Problem 6 ===================%
    \begin{problem}
        Extend the meet-in-the-middle attack to 3DES with 3 different keys and give the complexity for the attack.
    \end{problem}

    \begin{solution}
        
    \end{solution}

    %=================== Problem 7 ===================%
    \begin{problem}
        Provide a real-life example that a chosen-ciphertext attack is feasible.
    \end{problem}

    \begin{solution}
        
    \end{solution}

    %=================== Problem 8 ===================%
    \begin{problem}
        Comment (with an example) that the modified one-time-pad scheme II using a pseudorandom function cannot solve the problem of message integrity.
    \end{problem}

    \begin{solution}
        
    \end{solution}

    %=================== Problem 9 ===================%
    \begin{problem}
        {[Exercise 2.10 of the reference text]} Let G be a pseudorandom generator and define G'(s) to be the output of G truncated to n bits (where |s|=n). Prove that the function $F_k(s)=G'(k) \bigoplus x$ is not psedurandom.
    \end{problem}

    \begin{solution}
        
    \end{solution}

    %=================== Problem 10 ===================%
    \begin{problem}
        Consider the following public-key scheme. The public key is $(G, p, g, h)$ and the private key is $x$, generated exactly as in the Elgamal encryption scheme, where $h=g^x$. In order to encrypt a bit $b$, the sender does the following.

        \begin{enumerate}[label=(\alph*)]
            \item If $b=0$, then choose a random y from $Z_q^*$ and compute $c_1=g^y$ and $c_2=h^y$. The ciphertext if $(c_1, c_2)$.
            \item If $b=1$, then choose independent random $y$, $z$ from $Z_q^*$, compute $c_1=g^y$ and $c_2=g^z$, and set the ciphertext equal to $(c_1, c_2)$.
        \end{enumerate}

        Show that it is possible to decrypt efficiently given knowledge of $x$.
    \end{problem}

    \begin{solution}
        
    \end{solution}
\end{document}