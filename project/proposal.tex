\documentclass{article}

% \usepackage{nips_2018} % ready for submission
% \usepackage[preprint]{nips_2018} % compile a preprint version
% \usepackage[final]{nips_2018} % to compile a camera-ready version
% \usepackage[nonatbib]{nips_2018} % to avoid loading the natbib package
\usepackage[final, nonatbib]{nips_2018}

\usepackage[utf8]{inputenc} % allow utf-8 input
\usepackage[T1]{fontenc}    % use 8-bit T1 fonts
\usepackage{hyperref}       % hyperlinks
\usepackage{url}            % simple URL typesetting
\usepackage{booktabs}       % professional-quality tables
\usepackage{amsfonts}       % blackboard math symbols
\usepackage{microtype}      % microtypography

\title{Proposal: Caching Placement Decision with Deep Reinforcement Learning}

\author{
  HONG Yuncong \\ %\thanks{Use footnote for providing further information about author}
  Department of Computer Science \\
  The University of Hong Kong \\
  \texttt{ychong@cs.hku.hk} \\
  \And % Using \AND forces a line break at that point
  ZENG Qunsong \\
  Department of Electrical and Electronic Engineering \\
  The University of Hong Kong \\
  \texttt{qszeng@eee.hku.hk} \\
}

\begin{document}

\maketitle

\begin{abstract}
  Motivated by the developing of edge computing, there are urgin needs in efficiency network cache placement mechanism design in case to imrpove user experience and imrpove the network performance.
  The state-of-art cache placement design, use heuristic algorithm or simple online algorithm, trying to achieve better real-time performance. But with the development of deep reinfocement learning, we could consider and solve more complicated dynamic progamming online.
  In this paper, we come up with the ideas to use Deep Reinforcement Learning to overcome the challenge that we could not solve the MDP problem within accepted time.
  The evaluation on our implementation would be examined unser real data trace from Google which is widely used.
\end{abstract}

\section{Introduction}
  \textbf{(Identify Problem Area \& Its Importance)}
  With the development of the network application, the network caching is becoming one unigorable factor in the network functionality.
  Cache is a promising technology to improve the transmission efficiency of 

  One practice design is CDN (Content Deliver Network), which caches the contents from network server. The CDN servers keep updating with the original contents holder, and helps improve website access experience from users' side \cite{cloudflare}.

  The situation when desired files could be found in cache node is called a \textit{cache hit}. The percentage of accesses that result in cache hits is known as the **hit rate** or **hit ratio** of the cache.

  \textbf{(Reviewing Current Knowledge)}
  The problem existing in how to behave better in CDN is about caching.
  However, CDN are always facing with the problems to balance the communication and cost and hit rate from user.
  technologies are facing problems now. Because the network edge is becoming more and more complicated.

  \textbf{(Point out Research Gap)}
  We are going to use DRL to solve such problem, by observing on the \textit{bandwidth, resource usage} of the request link. By taking 

% \section{Related Works}
%   \label{review}
%   In this section

% The work in \cite{expert-cdn} cited, ... ...

% The work in \cite{expert-cdn} cited, ... ...

% The work in \cite{expert-cdn} cited, ... ...

\section{Proposal}
  In this section, we are going to fulfil the content required in the first proposal, which would includes the following part: Problem formulation, Machine Learning Algorithm to be used, and the evaluation method to examine our results.

  \subsection{Problem Formulation}
  The caching problem could be represented in the following format.
  As the all available file set $\mathcal{F}$, could be cached on the cache nodes set $\mathcal{S}$; one request from the user could be in the joint set $R = \mathcal{F} \times \mathcal{S}$.

  \subsection{Algorithm Selection}
    This paper is inspired by \cite{Pensieve}, which use an Actor-Critic Network to implement online ABR algorithm for better video streaming QoS.
    Which could achieves 0.2\% compared to the online optimal solution under certain situation.

    This algorithm based on Reinforcement Learning, which is well-covered in this book \cite{rl-intro}.
    The basic introduction could be given by:

  \subsection{Evaluation Plan}
  We are going to use the real production data from Google \cite{clusterdata:Reiss2011}.
  We are going to examine the data traces under our single criteria with , considering QoS (Quality of Service).

% \subsubsection*{Acknowledgments}
% Thanks for the inspiration from \textit{Pensieve} the paper that we could make up this model. Thanks for Pro TAN Haisheng's description on basic caching problem.

% \section*{References}
\bibliographystyle{IEEEtran}
\bibliography{proposal.bib}

\end{document}

%%%%%%%%%%%%%%%%%%%%%%%%%%%%%%%%%%%%%%%%%%%%%%%%%%%%%%%%%%%%%%%%%%%%%
% \begin{figure}
%   \centering
%   \fbox{\rule[-.5cm]{0cm}{4cm} \rule[-.5cm]{4cm}{0cm}}
%   \caption{Sample figure caption.}
% \end{figure}

% \ref{sample-table}
% \begin{table}
%   \caption{Sample table title}
%   \label{sample-table}
%   \centering
%   \begin{tabular}{lll}
%     \toprule
%     \multicolumn{2}{c}{Part}                   \\
%     \cmidrule(r){1-2}
%     Name     & Description     & Size ($\mu$m) \\
%     \midrule
%     Dendrite & Input terminal  & $\sim$100     \\
%     Axon     & Output terminal & $\sim$10      \\
%     Soma     & Cell body       & up to $10^6$  \\
%     \bottomrule
%   \end{tabular}
% \end{table}
%%%%%%%%%%%%%%%%%%%%%%%%%%%%%%%%%%%%%%%%%%%%%%%%%%%%%%%%%%%%%%%%%%%%%