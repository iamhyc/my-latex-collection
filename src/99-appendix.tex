\appendix
\section{Expression of Transition Matrix}
\label{append_1}
% In this section, we elaborate the method to approximate the Bellman's equation in Eqn (\ref{eqn:sp_0}).

%NOTE: transition matrix and vector for AP
\begin{definition}[Denotation of Transition on AP]
    Let ${\Gamma}^{(k)}_{m,j}(t,n)$ be the transition matrix of $\vec{R}^{(k)}_{m,j}(t,n)$ at the $n$-th time slot in the $t$-th broadcast interval ($\forall k\in\apSet, m\in\esSet, j\in\jSpace$).
    \begin{align}
        \Gamma^{(k)}_{m,j}(t,n) &\define
        \begin{bmatrix}
            1 & \bar{p}^{(k)}_{m,j,0} &                       &        &                           \\
            & 0                     & \bar{p}^{(k)}_{m,j,1} &        &                           \\
            &                       & \ddots                & \ddots &                           \\
            &                       &                       & \ddots & \bar{p}^{(k)}_{m,j,\Xi-1} \\
            &                       &                       &        & 0                         \\
        \end{bmatrix},
        % \begin{bmatrix}
        %     \theta^{(k)}_{m,j,0}(t,n) \\
        %     \theta^{(k)}_{m,j,1}(t,n) \\
        %     \vdots \\
        %     \vdots \\
        %     \theta^{(k)}_{m,j,\Xi}(t,n)
        % \end{bmatrix},
    \end{align}
    where $\bar{p}^{(k)}_{m,j,\xi}$ denote the probability of offloading ($\forall \xi=0,\dots,\Xi$)
    \begin{align}
        p^{(k)}_{m,j,\xi} &\define \Pr\{U^{(k)}_{m,j} < (\xi+1) | U^{(k)}_{m,j}>\xi\}
        \\
        \bar{p}^{(k)}_{m,j,\xi} &= 1 - p^{(k)}_{m,j,\xi}.
    \end{align}
    And we note that $\theta^{(k)}_{m,j,0}(t,n)$ is purely determined by the arrival process and dispatching policy of the $j$-th type of job on the $k$-th AP, i.e. $\theta^{(k)}_{m,j,0}(t,n) = \lambda_{k,j} I[\omega_{k,j}(t) = m]$, where $I[\cdot]$ is the indicator function.

    Let $\hat{\vecG{\Theta}}^{(k)}_{m,j}(t)$ denote the probability vector at the first time slot of the $t$-th broadcast interval and the state transition on AP between adjacent interval, i.e. from $\hat{\vecG{\Theta}}^{(k)}_{m,j}(t+1)$ to $\hat{\vecG{\Theta}}^{(k)}_{m,j}(t)$, is expressed as follows.
    \begin{align}
        \vecG{\Theta}^{(k)}_{m,j}(t, \mathcal{D}_{k}(t)) &= (\Gamma^{(k)}_{m,j})^{\mathcal{D}_{k}(t)} \times \hat{\vecG{\Theta}}_{m,j}(t),
        \nonumber\\
        \hat{\vecG{\Theta}}^{(k)}_{m,j}(t+1) &= (\Gamma^{(k)}_{m,j})^{N-\mathcal{D}_{k}(t)} \times \vecG{\Theta}^{(k)}_{m,j}(t, \mathcal{D}_{k}(t)).
    \end{align}
    % is composed of two-phase policy separated by $D_k(t)$, which is expressed as follows.
    % \begin{align}
    %     \vecG{\Theta}^{(k)}_{m,j}(t, \mathcal{D}_{k}(t)) &= (\Gamma^{(k)}_{m,j})^{\mathcal{D}_{k}(t)} \times \hat{\vecG{\Theta}}_{m,j}(t),
    %     \nonumber\\
    %     \hat{\vecG{\Theta}}^{(k)}_{m,j}(t+1) &= (\Gamma^{(k)}_{m,j})^{N-\mathcal{D}_{k}(t)} \times \vecG{\Theta}^{(k)}_{m,j}(t, \mathcal{D}_{k}(t)).
    % \end{align}
\end{definition}

%NOTE: small probability approximation
% The expression of transition matrix $P_{m,j}$ is more complex.
The transition happening on edge servers is affect by the arrival process from APs.
Hence, we first denote the offloading matrix $\bar{\Gamma}^{(k)}_{m,j}$ from each AP to the $m$-th edge server and the offloading number vector $\vecG{\rho}^{(k,+)}_{m,j}({t,n})$ as follows, respectively ($\forall k\in\apSet, m\in\esSet_{m}, j\in\jSpace$).
\begin{align}
    \bar{\Gamma}^{(k)}_{m,j}(t,n) &\define
    \begin{bmatrix}
        0 & p^{(k)}_{m,j,0} &                 &        &                     \\
        & 0               & p^{(k)}_{m,j,1} &        &                     \\
        &                 & \ddots          & \ddots &                     \\
        &                 &                 & \ddots & p^{(k)}_{m,j,\Xi-1} \\
        &                 &                 &        & 1                   \\
    \end{bmatrix},
    \\
    \vecG{\rho}^{(k,+)}_{m,j}({t,n}) &\define \bar{\Gamma}^{(k)}_{m,j} \times \vecG{\theta}^{(k)}_{m,j}({t,n}).
\end{align}
The combinations of all the offloading number vector for the $m$-th edge server from its \emph{potential AP set} would be unacceptable.
Thus we rewrite the arrival process on edge server with small probability approximation, i.e. there would be at most one job arriving in one time slot, with the probability as the expected arrival rate of the original distribution.
The explicit definition of the approximate arriving probability $\beta_{m,j}({t,n})$ is given as follows.
\begin{align}
    {\beta}_{m,j}({t,n}) &\define \sum_{k\in\apSet} \sum_{\xi=0,\dots,\Xi-1} \mathbb{E}[\vecG{\rho}^{(k,+)}_{m,j,\xi}({t,n})]
    \label{eqn_0}
\end{align}
\begin{lemma}[Small Probability Approximation]
    The probability distribution of $\sum_{k\in\apSet} \vecG{\rho}^{(k,+)}_{m,j}({t,n})$ could be approximated with a Bernoulli arrival process who is with the expected arrival rate denoted as ${\beta}_{m,j}({t,n})$.
\end{lemma}
\begin{proof}
    % We notice that the job arrival distribution ${\beta}_{m,j}(t)$ is given by $\mathcal{R}(t)$, and the departure rate in one slot is deterministic as $1/N$.
    % Thus the expectation of ${\beta}$ would be always far more smaller than $1$ as composed of all $K$ AP nodes.
    % We take approximation on ${\beta}$ as Bernoulli distribution in each time slot.
\end{proof}

%NOTE: transition matrix and vector for Edge Server
Thus we could obtain the denotation of transition matrix and probability vector for edge servers.
\begin{definition}[Denotation of Transition on Edge Server]
    Let $\vecG{\mu}_{m,j}(t,n)$ and $P_{m,j}$ denote the probability vector and transition matrix of $Q_{m,j}(t,n)$ at the $n$-th time slot in the $t$-th broadcast interval, respectively ($\forall m\in\esSet, j\in\jSpace$).
    \begin{align}
        \vecG{\mu}_{m,j}(t,n) \define [\Pr\{Q_{m,j}(t,n)=0\}, \dots, \Pr\{Q_{m,j}(t,n)=L_{max}\}].
    \end{align}

    Let $\hat{\vecG{\nu}}_{m,j}(t)$ denote the probability vector at the first time slot of the $t$-th broadcast interval.
    The time-variant transition matrix composed of multiple transition matrix $\mat{\hat{P}}_{m,j}(\beta({t,n}))$ in all the time slots in $i$-th interval as follows.
    \begin{align}
        \hat{\vecG{\nu}}({t,n+1}) &= \mat{\hat{P}}_{m,j}\Paren{\beta_{m,j}({t,n})} \hat{\vecG{\nu}}({t,n})
        % \label{eqn_3}
        \\
        \vecG{\nu}(t+1) &= \prod_{n=0,\dots,N-1} \mat{\hat{P}}_{m,j}\Paren{\beta_{m,j}({t,n})} \vecG{\nu}(t),
        \label{eqn_4}
    \end{align}
\end{definition}
% the probability distribution of $Q_{m,j}({t+i+1})$ is denoted by $\vecG{\nu}({t+i+1})$ which is obtained by calculating equation Enq. (\ref{eqn_0}) - Eqn (\ref{eqn_4}) ($\forall i=0,\dots,\frac{\Xi}{T}$);
