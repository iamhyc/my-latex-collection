\section{Decentralized Algorithm with Partial Information}
\label{sec:algorithm}

In this part, we shall introduce a novel approximation method to decouple the optimization on the right-hand-side of the Bellman's equations for arbitrary system state.
Specifically, the decoupling can be achieved via the following two steps:
\begin{enumerate}
    \item We first introduce a baseline policy and use its value function to approximate the value function of the optimal policy in Section \ref{subsec:baseline};
    \item Based on the approximate value function, an alternative action update algorithm, where a subset of APs are selected to update their dispatching action distributively in each broadcast interval, is proposed in Section \ref{subsec:ap_alg}.
    Moreover, the analytically performance bound is also derived.
\end{enumerate}

\subsection{Baseline Police and Approximate Value Function}
\label{subsec:baseline}
To alleviate the curse of dimensionality, we use the baseline policy with fixed dispatching action to approximate value function at the right-hand-side of the Bellman's equations.
Specifically, the baseline policy is elaborated below.

\begin{policy}[Baseline Policy]
    In the baseline policy $\Baseline$, each AP fixes the target processing edge server for each job type as the previous broadcast interval. Specifically,
    \begin{align}
        \Baseline_{t} &\define \Bracket{ \Pi_{t,1}, \dots, \Pi_{t,K} },
    \end{align}
    where 
    \begin{align}
        \Pi_{t,k} &\define \Brace{
            \omega_{k,j}(t) | \forall j\in\jSpace
        }, \forall k\in\apSet.
    \end{align}
\end{policy}

Thus, we shall approximate the value function of the optimal policy as
\begin{align}
    &V\Paren{\Stat(t+1)} \approx
    W^{\AP}_{\Baseline_{t}} \Paren{\Stat(t+1)} +
    W^{\ES}_{\Baseline_{t}} \Paren{\Stat(t+1)},
\end{align}
where $W^{\AP}_{\Baseline_{t}}(\Stat(t+1))$ and $W^{\ES}_{\Baseline_{t}}(\Stat(t+1))$ denote the cost raised by the APs and edge servers with the baseline policy $\Baseline_{t}$ and initial system state $\Stat(t+1)$, respectively.
The definitions are given below.
%FIXME: no baseline in the definition!
\begin{align}
    &W^{\dagger}_{\Baseline_{t}} \Paren{\Stat(t+1)} \define
    \nonumber\\
    &~~~~~~~~~~~\sum_{k\in\apSet}\sum_{m\in\esSet}\sum_{j\in\jSpace}\Bracket{
        \sum_{i=0}^{\infty} \gamma^{i+1} \Inorm{\vec{R}^{(k)}_{m,j}(t+i+1)}
    },
    \\    
    &W^{\ddagger}_{\Baseline_{t}}\Paren{\Stat(t+1)} \define
    \nonumber\\
    &~~~~~~~~~~~~~~~~\sum_{m\in\esSet}\sum_{j\in\jSpace}\Bracket{
        \sum_{i=0}^{\infty} \gamma^{i+1} Q_{m,j}(t+i+1)
    }.
\end{align}
\fixit{
    (DELETE)
    However, since the GSI and the global \brlatency~ is not obtainable in the edge computing system, we need to further decouple Problem \ref{problem_1} onto individual APs where only local OSI and \brlatency~is required to update the dispatching policy.
    Hence, in the following subsection \ref{subsec:ap_alg}, we propose a decentralized algorithm where the APs shall update their policies simultaneously in the form of disjoint clusters.
    The performance guarantee of the algorithm is also provided.
}

\subsection{The Decentralized Algorithm}
\label{subsec:ap_alg}
\comments{
    Although the optimal value function with the relaxation of GSI, denoted as $V$, has been approximated via the baseline policy in the previous part, it is still infeasible for all the APs to solve the right-hand-side of the Bellman's equations with OSI only.
    This is because \fixit{some information is not available}.
    Instead, it can be proved that it is feasible for part of the APs to update their dispatching actions distributively and achieve a better performance.
}
Hence, we first define the following sequence of AP subsets, where APs of each subset are selected to update dispatching strategy periodically, followed by the procedure of alternative update of dispatching actions.
\begin{definition}[Subsets of Periodic Strategy Update]
    Let $\mathcal{Y}_{1}, \dots, \mathcal{Y}_{N} \subseteq \ccSet$ be a sequence of subset, where each subset (say the $n$-th subset) satisfies the following constraints
    \begin{align}
        &\bigcup_{n=1,\dots,N} \mathcal{Y}_{n} = \apSet
        \\
        \ccSet_{y} \cap \ccSet_{y'} &=\emptyset, \forall y' \neq y \in \mathcal{Y}_{n}~(n=1,\dots,N).
    \end{align}
\end{definition}
\fixit{
    (Add one example with figure illustration).
}

% Hence, the proposed dispatching policy, based on the one-step policy iteration on the baseline policy, is defined as follows.
\fixit{
    Moreover, we define $\tilde{\mathbf{\Omega}}$ as the dispatching actions for the APs in $Y_{n}$.
}
Hence, in the $t$-th broadcast, the APs in the subset indexed with $n \equiv t \pmod{N} + 1$, i.e. $\mathcal{Y}_{n}$, shall update their dispatching actions.
Specifically, the optimization of \fixit{what dispatching actions of what} at the right-hand-side of the Bellman's equations can be rewritten as the following problem.
\delete{v14}{
    the dispatching policies of the APs outside the subset $\mathcal{Y}_{n}$ is given by
    \begin{align}
        \tilde{\Omega}_{y'}( \Stat_{y'}(t),\Delay_{y'}(t) ) \define \set{ \omega_{k,j}(t) | \forall j\in\jSpace }, \forall y'\notin\mathcal{Y}_{n}.
    \end{align}
    And the dispatching policies of the APs in the subset $\mathcal{Y}_{n}$ is given by solving the following Problem \ref{problem_2}.
    % The optimization problem at right-hand side of approximate Bellman's Equation is given as follows.
}

\begin{problem}[]
    In the $t$-th broadcast interval, the $y$-th AP in the set $\mathcal{Y}_{n}$ ($n \equiv t \pmod{N} + 1$) shall solve the following problem to update its policy.
    {\small
    \begin{align}
        \min_{ \tilde{\Omega}_{y}(\Stat_{y}(t),\Delay_{y}(t)) }
        &\sum_{\Stat(t+1)} \Pr\Brace{
            \Stat(t+1) | \Stat(t), \tilde{\Omega}_{y}(\Stat_{y}(t),\Delay_{y}(t))
        } \times
        \nonumber\\
        &\Bracket{
            W^{\AP}_{\Baseline(\Stat(t))} \Paren{\Stat(t+1)} +
            W^{\ES}_{\Baseline(\Stat(t))} \Paren{\Stat(t+1)}
        },
    \end{align}
    }
    \label{problem_2}
\end{problem}
\fixit{where the $\tilde{\Omega}$ is how defined.}

Moreover, we have the following conclusion on the decomposition of Problem \ref{problem_2}.
\begin{lemma}
    \fixit{(decomposition lemma by split the expression basedon OSI).}    
\end{lemma}

\fixit{Based on the above lemma, we could rewrite the Problem \ref{problem_2} as the following reduced form.}
\begin{problem}[]
    In the $t$-th broadcast interval, the $y$-th AP in the set $\mathcal{Y}_{n}$ ($n \equiv t \pmod{N} + 1$) shall solve the following problem based on its OSI and local \brlatency~information
    {\small
    \begin{align}
        &\min_{\tilde{\Omega}_{y}(\Stat_{y}(t),\Delay_{y}(t))}
        \mathbb{E}_{\set{\Stat(t+1), \tilde{\Omega}_{y}(\Stat_{y}(t),\Delay_{y}(t))}}
        \nonumber\\
        &~~~~~~~~~~~\Bracket{
            W^{\AP}_{ \Pi_{k}(\Stat_{k}(t)) }\Paren{\Stat_{k}(t+1)} +
            W^{\ES}_{ \Pi_{k}(\Stat_{k}(t)) }\Paren{\Stat_{k}(t+1)}
        }.
        \label{eqn:partial}
    \end{align}
    }
    \label{problem_3}
\end{problem}

\fixit{
    (SOLUTION ALGORITHM to PROBLEM 3)
    The problem is easy to solve by brutally evalute the value function for all $m\in\esSet_{k}$ for the $k$-th AP ($\forall k\in\ccSet$).
    \begin{remark}[Complexity Analysis]
        The complexity of the above solution to the algorithm is $O(J(KM+M))$.
    \end{remark}
}

\fixit{
    Specifically, the definitions of symbols in the remaining parts is given in the Appendix \ref{apped_1} and the calculation of Eqn (\ref{eqn:partial}) is elaborated below.
    The expected value function $W^{\AP}_{\Pi_{k}(\Stat_{k}(t))}(\Stat_{k}(t+1))$ is easily obtained by evaluating the following equation
    \begin{align}
        W^{\AP}_{\Pi_{k}(\Stat_{k}(t))}\Paren{\Stat(t+1)} &= \sum_{j\in\jSpace}\sum_{k\in\ccSet_{k}}\sum_{m\in\esSet_{k}}
        \nonumber\\
        &\Inorm{
            \Paren{ 1 - \gamma ({\Gamma}^{(k)}_{m,j})^{N} }^{-1} \hat{\vecG{\Theta}}^{(k)}_{m,j}(t+1)
        }.
        \label{w_ap}
    \end{align}
    % where $\hat{\Gamma}^{(k)}_{m,j} \define (\Gamma^{(k)}_{m,j})^{N}$.

    The expression for expected value function is little more complex compared to Eqn (\ref{w_ap}).
    It is affected with both arrival process under dispatching policy and last queue state.
    However, we notice that the arrival process would be stationary after the maximum uploading time under the stationary baseline policy and the relationship between APs and edge server could be decoupled.
    Hence, the expected value function $W^{\ES}_{\Pi_{k}(\Stat_{k}(t))}(\Stat_{k}(t+1))$ is obtained by the following equation.
    {\small
    \begin{align}
        &W^{\ES}_{\Pi_{k}(\Stat_{k}(t))}\Paren{\Stat_{k}(t+1)}
        = \sum_{j\in\jSpace}\sum_{m\in\esSet_{k}}\sum_{i=0,\dots,\frac{\Xi}{T}} \gamma^{i} \mathbb{E}[ Q_{m,j}({t+i+1}) ]
        \nonumber\\
        &~~~~~~~~~~~~~~~~~~~+ \gamma^{\frac{\Xi}{T}} \Paren{ \mat{I} - \gamma \hat{\mat{P}}_{m,j}(\tilde{\beta}_{m,j}) }^{-1} \vecG{\nu}({t+\frac{\Xi}{T}+1}) \vec{g}',
    \end{align}   
    }
    where the $i$-th element of vector $\vec{g}$ denotes the cost of server as $Q_{m,j}(t)$, and
    \begin{align}
        \hat{\mat{P}}_{m,j}(\tilde{\beta}_{m,j}) \define \prod_{n=0,\dots,N-1} \mat{P}_{m,j}(\tilde{\beta}_{m,j}),
    \end{align}
    % the probability distribution of $Q_{m,j}({t+i+1})$ is denoted by $\vecG{\nu}({t+i+1})$ which is obtained by calculating equation Enq. (\ref{eqn_0}) - Eqn (\ref{eqn_4}) ($\forall i=0,\dots,\frac{\Xi}{T}$);
    where $\tilde{\beta}_{m,j}$ is the arrival distribution under baseline policy $\Pi(t)$ (on $m$-th ES with type $j$ jobs)
    \begin{align}
        \tilde{\beta}_{m,j} &\define \sum_{k\in\apSet} \tilde{\lambda}^{(k)}_{m,j} \times \sum_{\xi=0,\dots,\Xi-1} \Pr\{ \xi<U_{k,m}\leq(\xi+1) \}
            \nonumber\\
        ~~~~&= \sum_{k\in\apSet} \tilde{\lambda}^{(k)}_{m,j}
    \end{align}
}

\subsection{Algorithm?}
The details of the procedure is elaborated as follows.
\begin{enumerate}
    \item Initialize the dispatching policy of the $k$-th AP ($\forall k\in\apSet$) with some heuristic policy as $\Policy_{k}(\Stat_{k}(0), \Delay_{k}(0))$;
    \item At the $1$-st broadcast time slot when $t=0$, the APs in set $\mathcal{Y}_{1}$ and the edge servers in the corresponding \emph{candidate server set} of APs in $\mathcal{Y}_{1}$, shall broadcast their LSI (including the heuristic policies);
    \item Each AP (say the $y_1$-th AP) in set $\mathcal{Y}_{1}$ would receive the OSI at the $D_{y_1}(1)$ time slots in the $1$-st broadcast interval, and then it updates its policy $\Omega_{y_1}(\Stat_{y_1}(1))$ by solving Eqn (\ref{eqn:partial}); The APs in set $\mathcal{Y}_{1}$ would keep with this policy until they receive the OSI again;
    % \item At the $2$-nd broadcast time slot when $t=1$, the APs in set $\mathcal{Y}_{2}$ would receive the broadcast OSI;
    \item Similarly, in the $t$-th broadcast interval the APs in subset indexed with $n \equiv (t + 1)\mod{N}$ would receive the OSI, and then update their policy $\Omega_{y_n}(t+1)$ ($\forall y_n\in\mathcal{Y}_{n}$) by solving Eqn (\ref{eqn:partial}).
    \item Step 4) is periodically repeated with respect to the broadcast interval.
\end{enumerate}

% \begin{remark}[Complexity Analysis]
%     The computation complexity of solving Problem \ref{problem_3} is $O(J())$

%     % The computation complexity of solving Problem \ref{problem_3} is $O(J(c_k M_k+M_k))$ where $c_k, M_k$ is the number of APs in the \emph{conflict AP set} and number of edge servers in the \emph{potential server set} of the $k$-th AP, respectively.
%     Overall, the computation complexity of the algorithm in one period is $O(J(KM+M))$.
% \end{remark}

\comments{
    Finally, we have the following conclusion on the performance of the above proposed algorithm.
}
\begin{lemma}[Performance Guarantee]
    % The optimized policy solved as $\Baseline(t)$ for next stage, is better than $\Baseline(t-1), \dots, \Baseline(1)$ when considering the \emph{approximated value function} defined above.
    The performance of the series of the baseline policies is upper bounded by the optimal solution $\Policy^*$ when considering the original Bellman's equation.
    $$
        V_{\Omega^*}\Paren{\Stat(t)}
        \leq W_{\tilde{\Policy}(\Stat(t))}\Paren{\Stat(t)}
        \leq W_{\tilde{\Policy}(\Stat(t-1))}\Paren{\Stat(t)},
        \forall \Stat(t)
    $$
\end{lemma}
\begin{proof}
    
\end{proof}

%----------------------------------------------------------------------------------------%
\delete{v11}{
    % [\IF, \ENDIF], [\FOR, \TO, \ENDFOR], [\WHILE, \ENDWHILE], \STATE, \AND, \TRUE
    % \begin{algorithm}[H]
    %     \caption{Online Iterative Policy Improvement Algorithm}
    %     \begin{algorithmic}[1]
    %         \STATE $t = 0$
    %         \FOR{$t = 1,2,\dots$}
    %             \STATE Evaluate $\Omega_0$ in \textbf{P1} according to Eqn (\ref{sp1})
    %             \FOR{$k \in \mathcal{K}$}
    %                 \STATE fix policy $\vec{\Omega}^{(k)}(t) \forall k' < k$
    %                 \STATE Evaluate $k$-th AP Local Policy $\tilde{\Omega}_k$ in \textbf{Pk} according to Eqn (\ref{sp2})
    %             \ENDFOR
    %         \ENDFOR
    %     \end{algorithmic}
    % \end{algorithm}
}
%----------------------------------------------------------------------------------------%