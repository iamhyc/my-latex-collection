\section{Decentralized Algorithm with Partial Information}
\label{sec:algorithm}

In this section, we shall introduce a novel approximation method to decouple the centralized optimization on the RHS of the Bellman's equations to each AP for arbitrary system state.
Specifically, the decoupling can be achieved via the following two steps:
\begin{enumerate}
    \item We first introduce a baseline policy, use its value function to approximate the value function of the optimal policy $\Policy^*$, and derive the analytical expression of approximate value function in Section \ref{subsec:baseline}.
    \item Based on the approximate value function, an alternative action update algorithm, where a subset of APs are selected to update their dispatching action distributively in each broadcast interval, is proposed in Section \ref{subsec:ap_alg}.
    Moreover, the analytically performance bound is also derived.
\end{enumerate}

\subsection{Baseline Police and Approximate Value Function}
\label{subsec:baseline}
To alleviate the curse of dimensionality, we first use the baseline policy with fixed dispatching action to approximate value function at the RHS of the Bellman's equations in equation (\ref{eqn:val_f}).
Specifically, the baseline policy is elaborated below.

\begin{policy}[Baseline Policy]
    In the baseline policy $\Baseline$, each AP fixes the target processing edge server for each job type as the previous broadcast interval. Specifically, in the $t$-th broadcast interval,
    \begin{align}
        \Baseline(\Stat(t)) &\define \Bracket{ \Pi_{1}(\Stat_{1}(t)), \dots, \Pi_{K}(\Stat_{K}(t)) },
    \end{align}
    where 
    \begin{align}
        \Pi_{k}(\Stat_{k}(t)) &\define \Brace{
            \omega_{k,j}(t) | \forall j\in\jSpace
        }, \forall k\in\apSet.
    \end{align}
\end{policy}

Let $W_{\Baseline}(\cdot)$ be the value function of the baseline policy, and we shall approximate the value function of the optimal policy $V(\cdot)$ via $\Baseline$, i.e.
\begin{align}
    V\Paren{\Stat(t+1)} &\approx W_{\Baseline}\Paren{\Stat(t+1)}
    \nonumber\\
    &= \sum_{m\in\esSet,j\in\jSpace}\Brace{
        \sum_{k\in\apSet} \tilde{W}^{\AP}_{k,m,j}(\Stat(t+1))
        \nonumber\\
        &~~~~~~~~~~~~~~~~~~~~~~+\tilde{W}^{\ES}_{m,j}(\Stat(t+1))
    },
\end{align}
where $\tilde{W}^{\AP}_{k,m,j}(\Stat(t+1))$ denotes the cost raised by the type-$j$ job which is transmitting from the $k$-th AP to the $m$-th edge server with the baseline policy $\Baseline$ and initial system state $\Stat(t+1)$, and $\tilde{W}^{\ES}_{m,j}(\Stat(t+1))$ denotes the cost raised by the type-$j$ job on the $m$-th server.
Their definitions are given below.
{\small
\begin{align}
    \tilde{W}^{\AP}_{k,m,j} \Paren{\Stat(t+1)} &\define
        \sum_{i=0}^{\infty} \gamma^{i+1} \mathbb{E}^{\Baseline}\Bracket{
            \Inorm{\vec{R}^{(k)}_{m,j}(t+i+1)}
        },
    \\    
    \tilde{W}^{\ES}_{m,j} \Paren{\Stat(t+1)} &\define
        \sum_{i=0}^{\infty} \gamma^{i+1} \mathbb{E}^{\Baseline}\Bracket{
            Q_{m,j}(t+i+1) +
            \nonumber\\
            &~~~~~~~~~~\beta I[Q_{m,j}(t+i+1) = L_{max}]
        }.
\end{align}
}

Moreover, the explicit expression of $\tilde{W}^{\AP}_{k,m,j}(\Stat(t+1))$ and $\tilde{W}^{\ES}_{m,j}(\Stat(t+1))$ are derived in the following lemmas, respectively.

\begin{lemma}[Analytical Expression of $\tilde{W}^{\AP}_{k,m,j}$]
    \label{lemma:w_ap}
    \begin{align}
        &\tilde{W}^{\AP}_{k,m,j}\Paren{\Stat(t+1)} =
        \Inorm{
            \Bracket{
                \vecG{\Theta}^{(k, \Baseline)}_{m,j}(t+1)
            }'
            \Bracket{
                \mat{I} - \gamma \Gamma^{(k)}_{m,j}
            }^{-1}
        },
        \label{w_ap}
    \end{align}
    where $\vecG{\Theta}^{(k, \Baseline)}_{m,j}(t)$ and $\Gamma^{(k)}_{m,j}$ are defined below.
    \begin{itemize}
        % \item $\vecG{\Theta}^{(k, \Baseline)}_{m,j}(t) \in \mathbb{R}^{(\Xi+1) \times 1}$ denotes the probability vector under baseline $\Baseline$
        \item $\vecG{\Theta}^{(k)}_{m,j}(t) \define \Bracket{ \theta^{(k)}_{m,j,0}(t), \theta^{(k)}_{m,j,1}(t), \dots, \theta^{(k)}_{m,j,\Xi}(t) }$,
        where 
        \begin{align}
            \theta^{(k)}_{m,j,\xi}(t) \define 
            \begin{cases}
                \lambda_{k,j} I[\omega_{k,j}(t)=m], & \xi=0
                \\
                \Pr\{R^{(k)}_{m,j,\xi}(t,0) = 1\}, & \xi=1,\dots,\Xi
            \end{cases}
        \end{align}
        % \item $\Gamma^{(k)}_{m,j} \define (\hat{\Gamma}^{(k)}_{m,j})^N$ where
        \item
        \begin{align}
            \Gamma^{(k)}_{m,j} &\define
            \begin{bmatrix}
                1 & \bar{p}^{(k)}_{m,j,0} &                       &        &                           \\
                  & 0                     & \bar{p}^{(k)}_{m,j,1} &        &                           \\
                  &                       & \ddots                & \ddots &                           \\
                  &                       &                       & \ddots & \bar{p}^{(k)}_{m,j,\Xi-1} \\
                  &                       &                       &        & 0                         \\
            \end{bmatrix}^{N}
        \end{align}
        and $\bar{p}^{(k)}_{m,j,\xi} \define 1 - \Pr\{U^{(k)}_{m,j} < \xi+1 | U^{(k)}_{m,j}>\xi\}$.
    \end{itemize}
    % \begin{align}
    %     \vec{\Theta}_{m,j}^{(k)}(t+1) = {\Gamma}^{(k)}_{m,j}(\Pi_{t,k}) \times \vec{\Theta}_{m,j}^{(k)}(t).
    % \end{align}
\end{lemma}
\begin{proof}
    Please refer to Appendix \ref{append_1}.
\end{proof}

\begin{lemma}[Analytical Expression of $\tilde{W}^{\ES}_{m,j}$]
    \label{lemma:w_es}
    {\small
    \begin{align}
        &\tilde{W}^{\ES}_{m,j}\Paren{\Stat(t+1)}
        = \sum_{i=0,\dots,\frac{\Xi}{T}} \gamma^{i} \mathbb{E}^{\Baseline}[ Q_{m,j}({t+i+1}) ]
        \nonumber\\
        &~~~~~~~~~~~~+ \gamma^{\frac{\Xi}{T}} 
        \vecG{\nu}({t+\frac{\Xi}{T}+1})
        \Paren{
            \mat{I} - \gamma \mat{P}_{m,j}(\beta_{m,j}(t))
        }^{-1} \vec{g}',
        \label{w_es}
    \end{align}   
    }
    where $\vecG{\nu}_{m,j}(t)$, $\mat{P}_{m,j}(\beta_{m,j}(t))$, $\beta_{m,j}(t)$ and $\vec{g}$ are defined below.
    \begin{itemize}
        \item {\small
        $\vecG{\nu}_{m,j}(t) \define [\Pr\{Q_{m,j}(t)=0\}, \dots, \Pr\{Q_{m,j}(t)=L_{max}\}]$
        }.

        \item $\vec{g} \in \mathbb{R}^{(L_{max}+1) \times 1}$ whose $i$-th element is
        \begin{align}
            [\vec{g}]_{i} = 
            \begin{cases}
                i, & i=0,1,\dots,L_{max}-1
                \\
                L_{max}+\beta, & \text{otherwise}
            \end{cases}.
        \end{align}

        \item $\mat{P}_{m,j}(\beta_{m,j}(t)) \in \mathbb{R}^{(L_{max}+1) \times (L_{max}+1)}$ denotes the transition matrix given the average job arrival rate $\beta_{m,j}(t)$ under baseline policy $\Baseline(\Stat(t))$ where
        \begin{align}
            \beta_{m,j}(t) = \sum_{k\in\apSet} \lambda_{k,j} I[\omega_{k,j}(t)=m].
        \end{align}
        The entries of the transition matrix $\mat{P}_{m,j}$ are provided in Appendix \ref{append_1}.
    \end{itemize}   
\end{lemma}
\begin{proof}
    Please refer to Appendix \ref{append_1}.
\end{proof}

\subsection{The Decentralized Algorithm}
\label{subsec:ap_alg}
Although the optimal value function has been approximated via the baseline policy in the previous part, it is still infeasible for all the APs to solve the RHS of the Bellmans equations in a distributive manner with OSI only.
This is because the evaluation of equation (\ref{w_ap}) and (\ref{w_es}) requires the knowledge of GSI at each AP.
Instead, it is feasible for part of APs to update their dispatching actions distributively and achieve a better performance compared with baseline policy.
Hence, we first define the following sequence of AP subsets, where each subset are selected to update dispatching actions periodically.
\begin{definition}[Subsets of Periodic Actions Update]
    Let $\mathcal{Y}_{1}, \dots, \mathcal{Y}_{N} \subseteq \ccSet$ be a sequence of subset, where each subset satisfies the following constraints
    \begin{align}
        &\bigcup_{n=0,\dots,N-1} \mathcal{Y}_{n} = \apSet
        \\
        \esSet_{y} \cap \esSet_{y'} &=\emptyset, y' \neq y~(\forall y',y \in \mathcal{Y}_{n}).
    \end{align}
\end{definition}
For example, as illustrated in Fig.\ref{fig:conflict}, the AP set $\apSet$ could be decomposed of two subsets as $\set{1,3}$ and $\set{2}$.
Hence, in the $t$-th broadcast interval, the APs in the subset indexed with $n \define t \pmod{N}$ update their dispatching actions, while the other APs keep the same dispatching actions as the previous broadcast interval.
Hence, let
\begin{align}
    \tilde{\mathcal{A}}(t) \define \Brace{
        \tilde{\mathcal{A}}_{y}(t) \define \set{ \tilde{\omega}_{y,j}(t)\in \esSet_{y}|\forall j\in\jSpace } | \forall y\in\mathcal{Y}_{n}
    }
\end{align}
be the aggregation of dispatching actions for the APs in the subset $\mathcal{Y}_{n}$, and
\begin{align}
    \hat{\mathcal{A}}(t) \define \set{\Pi_{y}(\Stat(t)) | \forall y\notin\mathcal{Y}_{n}}
\end{align}
be the aggregation of dispatching actions of the rest APs.
In the $t$-th broadcast interval, the optimization of $\tilde{\mathcal{A}}_{y}(t)$ ($\forall y\in\mathcal{Y}_{n}$) at the RHS of the Bellman's equations can be rewritten as the following problem.
{\small
\begin{align}
    \textbf{P2:}~
    \min_{ \tilde{\mathcal{A}}(t) }
    &\sum_{\Stat(t+1)} \Pr\Brace{
        \Stat(t+1) | \Stat(t), \tilde{\mathcal{A}}(t), \hat{\mathcal{A}}(t)
    } \cdot W_{\Baseline}\Paren{\Stat(t+1)},
\end{align}
}

Moreover, we have the following conclusion on the decomposition of P2.
\begin{lemma}[]
    The optimization problem in P2 can be equivalently decoupled into the following local optimization problem.
    {\small
    \begin{align}
        \textbf{P3:}~
        \min_{ \tilde{\mathcal{A}}_{y}(t) }
        \mathbb{E}_{\set{ \Stat(t+1), \tilde{\mathcal{A}}_{y}(t) }}
        &\sum_{j\in\jSpace,m\in\esSet_{y}} \Brace{
            \tilde{W}^{\AP}_{y,m,j}\Paren{\Stat(t+1)}
            \nonumber\\
            &~~~~~~~~~~~~~~~~~+\tilde{W}^{\ES}_{m,j}\Paren{\Stat(t+1)}
        }.
        % \nonumber\\
        % &~~~~~~~~~~~\Bracket{
        %     W^{\AP}_{ \Baseline }\Paren{\Stat_{k}(t+1)} +
        %     W^{\ES}_{ \Baseline }\Paren{\Stat_{k}(t+1)}
        % }.
        \label{eqn:partial}
    \end{align}
    }
\end{lemma}
% \begin{proof}
%     We could simply reduce the expression of equation (\ref{w_ap}) and equation (\ref{w_es}) on the RHS of the Bellman's equations based on local OSI, while GSI is not necessary.
%     because the update policy would only change the cost raised on local AP and the corresponding \emph{candidate server set}.
% \end{proof}

The optimization of $\tilde{\mathcal{A}}_{y}(t)$ in P3 could be applied by linear search over the solution space of size $|\esSet_{y}|$.
The computational complexity of solving P3 with linear search is $O(J(KM+M))$ for evaluating equation (\ref{eqn:partial}).

As a result, the overall algorithm of job dispatching is elaborated as follows.
% [\IF, \ENDIF], [\FOR, \TO, \ENDFOR], [\WHILE, \ENDWHILE], \STATE, \AND, \TRUE
\begin{algorithm}[!]
    \caption{Online Alternative Actions Update Algorithm}\label{alg_1}
    \DontPrintSemicolon % Some LaTeX compilers require you to use \dontprintsemicolon instead
    % \KwIn{$\Stat(t), \Delay(t)$}
    % \KwOut{$\tilde{\mathcal{A}}(t)$}

    Initialize $\tilde{\mathcal{A}}(0),\hat{\mathcal{A}}(0)$ with heuristic dispatching actions.\;
    \For{$t=0,1,2,\dots$}{
        $n \gets t \pmod{N}$\;
        \For{$y \in \mathcal{Y}_{n}$}{
            Observe $\Stat_{y}(t)$ after $\mathcal{D}_{y}(t)$\;
            $\tilde{\mathcal{A}}_{y}(t+1) \gets$ solving Problem P3 with $\mathcal{D}_{y}(t), \Stat_{y}(t)$\;
        }
        % \Return $\tilde{\mathcal{A}}(t+1)$\;
    }
\end{algorithm}
% At the $1$-st broadcast time slot when $t=0$, the APs in set $\mathcal{Y}_{1}$ and the edge servers in the corresponding \emph{candidate server set}s of APs in $\mathcal{Y}_{1}$, should broadcast their LSI (including the heuristic actions)

Finally, we have the following conclusion on the performance of the above proposed algorithm.
\begin{lemma}[Performance Guarantee]
    % The optimized policy solved as $\Baseline(t)$ for next stage, is better than $\Baseline(t-1), \dots, \Baseline(1)$ when considering the \emph{approximated value function} defined above.
    Let $W_{\tilde{\Policy}}(\cdot)$ be the value function of the policy $\tilde{\Policy}$ where
    $\tilde{\Policy}(\Stat(t)) \define \tilde{\mathcal{A}}(t) \cup \hat{\mathcal{A}}(t)$,
    \begin{align}
        W_{\tilde{\Policy}}(\Stat) \define
        \sum_{t=1}^{\infty} \gamma^{t-1} \mathbb{E}^{\tilde{\Policy}} \Bracket{
            g\Paren{\Stat(t), \tilde{\Policy}(\Stat(t))} | \Stat(1)=\Stat
        },
    \end{align}
    % The performance of the baseline policies is upper bounded by the optimal solution $\Policy^*$ when considering the original Bellman's equations.
    we have
    \begin{align}
        V_{\Policy^*}\Paren{\Stat(t)}
        \leq W_{\tilde{\Policy}}\Paren{\Stat(t)}
        \leq W_{\Baseline}\Paren{\Stat(t)},
        \forall \Stat(t).
    \end{align}
\end{lemma}
\begin{proof}
    \fixit{
        (TODO)
    }
\end{proof}

%----------------------------------------------------------------------------------------%
\delete{v14}{
    Moreover, we notice that the transition function in equation (\ref{eqn:sp_0}) could be rewrite in the following form.
    \begin{align}
        & \Pr\Brace{ \Stat_{k}(t+1)|\Stat_{k}(t), \Omega_{k}(\Stat_{k}(t)) }
        \nonumber\\
        =& \prod_{j\in\jSpace} \Brace{
            \prod_{k'\in\ccSet_{k}}\prod_{m\in\esSet_{k}} \Pr\big\{
                \vec{R}^{(k')}_{m,j}(t+1) | \vec{R}^{(k')}_{m,j}(t), \Omega_{k}(\Stat_{k}(t))
            \big\}
            \nonumber\\
            &\times \prod_{m\in\esSet_{k}} \Pr\big\{
                Q_{m,j}(t+1) | Q_{m,j}(t), \Omega_{k}(\Stat_{k}(t))
            \big\}
        },
    \end{align}
    where the transition function is decoupled into two parts of state transition on APs and edge servers, respectively.

    However, since the GSI and the global \brlatency~ is not obtainable in the edge computing system, we need to further decouple Problem P1 onto individual APs where only local OSI and \brlatency~is required to update the dispatching policy.
    Hence, in the following subsection \ref{subsec:ap_alg}, we propose a decentralized algorithm where the APs shall update their policies simultaneously in the form of disjoint clusters.
    % The performance guarantee of the algorithm is also provided.
}
%----------------------------------------------------------------------------------------%