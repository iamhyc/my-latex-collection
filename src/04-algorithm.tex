\section{Decentralized Algorithm with Partial Information}
\label{sec:algorithm}

In this part, we shall introduce a novel approximation method to decouple the optimization on the right-hand-side of the Bellman's equations for arbitrary system state.
Specifically, the decoupling can be achieved via the following two steps:
\begin{enumerate}
    \item we first introduce a baseline policy and use its value function to approximate the value function of the optimal policy;
    \item based on the approximate value function, only a {subset of APs} are scheduled for update the dispatching action in each broadcast interval, \comments{and the decentralized algorithm is proposed}.
\end{enumerate}
\comments{
    Step 1 shall be introduced in subsection \ref{subsec:baseline}, and then step 2 shall be introduced in subsection \ref{subsec:ap_alg}.
}

\subsection{Baseline Police and Approximate Value Function}
\label{subsec:baseline}
\accept{
    To alleviate the curse of dimensiionality, we use fixed policy to approximate the right-hand-side of the Bellman's equations.
    Specifically, the baseline policy used in $\Stat(t+1)$ is the policy taken in the $t$-th interval. %i.e. $\Omega(\Stat(t), \Delay(t))$.
    The explicit definition of baseline policy is given below.
}
\begin{definition}[Baseline Policy]
    In the baseline policy $\Baseline$, each AP fixes the target processing edge server for each job type as the previous broadcast interval. Specifically,
    \begin{align}
        \Baseline(\Stat(t)) &\define \Bracket{ \Pi_{1}(\Stat_{1}(t)), \dots, \Pi_{K}(\Stat_{K}(t)) },
    \end{align}
    where 
    \begin{align}
        \Pi_{k}(\Stat_{k}(t)) &\define \Brace{
            \omega_{k,j}(t) | \forall j\in\jSpace
        }, \forall k\in\apSet.
    \end{align}
\end{definition}

Thus, we shall approximate the value function of the optimal policy as
\begin{align}
    &V\Paren{\Stat(t+1)} \approx
    \nonumber\\
    &~~~~~~W^{\dagger}_{\Baseline(\Stat(t))} \Paren{\mathcal{R}(t+1)}
    + W^{\ddagger}_{\Baseline(\Stat(t))}\Paren{\mathcal{Q}(t+1)},
\end{align}
where
%FIXME: no baseline in the definition!
\begin{align}
    &W^{\dagger}_{\Baseline(\Stat(t))} \Paren{\mathcal{R}(t+1)} \define
    \nonumber\\
    &~~~~~~~~~~~\sum_{k\in\apSet}\sum_{m\in\esSet}\sum_{j\in\jSpace}\Bracket{
        \sum_{i=0}^{\infty} \gamma^{i+1} \Inorm{\vec{R}^{(k)}_{m,j}(t+i+1)}
    },
    \\    
    &W^{\ddagger}_{\Baseline(\Stat(t))}\Paren{\mathcal{Q}(t+1)} \define
    \nonumber\\
    &~~~~~~~~~~~~~~~~\sum_{m\in\esSet}\sum_{j\in\jSpace}\Bracket{
        \sum_{i=0}^{\infty} \gamma^{i+1} Q_{m,j}(t+i+1)
    }.
\end{align}
\comments{
    However, since the GSI and the global \brlatency~ is not obtainable in the edge computing system, we need to further decouple Problem \ref{problem_1} onto individual APs where only local OSI and \brlatency~is required to update the dispatching policy.
    Hence, in the following subsection \ref{subsec:ap_alg}, we propose a decentralized algorithm where the APs shall update their policies simultaneously in the form of disjoint clusters.
    The performance guarantee of the algorithm is also provided.
}

\subsection{The Decentralized Algorithm}
\label{subsec:ap_alg}
We firstly define the following sequence of AP subsets, where APs of each subset are selected to update dispatching strategy periodically.
\begin{definition}[Subsets of Periodic Strategy Update]
    Let $\mathcal{Y}_{1}, \dots, \mathcal{Y}_{N} \subseteq \ccSet$ be a sequence of subset, where each subset (say the $n$-th subset) satisfies the following constraints
    \begin{align}
        &\bigcup_{n=1,\dots,N} \mathcal{Y}_{n} = \apSet
        \\
        \ccSet_{y} \cap \ccSet_{y'} &=\emptyset, \forall y' \neq y \in \mathcal{Y}_{n}~(n=1,\dots,N).
    \end{align}
\end{definition}

% Hence, the proposed dispatching policy, based on the one-step policy iteration on the baseline policy, is defined as follows.
Hence, in the $t$-th broadcast, the APs in the subset indexed with $n \equiv t \pmod{N} + 1$, i.e. $\mathcal{Y}_{n}$ shall update their policies.
Specifically, the dispatching policies of the APs outside the subset $\mathcal{Y}_{n}$ is given by
\begin{align}
    \tilde{\Omega}_{y'}( \Stat_{y'}(t) ) \define \set{ \omega_{k,j}(t) | \forall j\in\jSpace }, \forall y'\notin\mathcal{Y}_{n}.
\end{align}
And the dispatching policies of the APs in the subset $\mathcal{Y}_{n}$ is given by solving the following Problem \ref{problem_2}.

\begin{problem}[]
    In the $t$-th broadcast interval, the $y$-th AP in the set $\mathcal{Y}_{n}$ ($n \equiv t \pmod{N} + 1$) shall solve the following problem to update its policy.
    {\small
    \begin{align}
        &\min_{ \tilde{\Omega}_{y}(\Stat_{y}(t)) }
        \sum_{\Stat(t+1)} \Pr\Brace{
            \Stat(t+1) | \Stat(t), \tilde{\Omega}_{y}(\Stat_{y}(t))
        } \times
        \nonumber\\
        &~~~~~~~~~~~\Bracket{
            W^{\dagger}_{\Baseline(\Stat(t))} \Paren{\mathcal{R}(t+1)} + W^{\ddagger}_{\Baseline(\Stat(t))}\Paren{\mathcal{Q}(t+1)}
        }.
    \end{align}
    }
    \label{problem_2}
\end{problem}
Moreover, we notice that the system information in $\Stat(t)$ and $\Stat(t+1)$ is not obtainable and also useless when solving Problem \ref{problem_2}.
Hence, we could further rewrite the problem in the following reduced form.
% The optimization problem at right-hand side of approximate Bellman's Equation is given as follows.
\begin{problem}[]
    In the $t$-th broadcast interval, the $y$-th AP in the set $\mathcal{Y}_{n}$ ($n \equiv t \pmod{N} + 1$) shall solve the following problem based on its OSI and local \brlatency
    {\small
    \begin{align}
        \min_{\tilde{\Omega}_{y}(\Stat_{y}(t))}
        &\mathbb{E}_{\set{\Stat(t+1), \tilde{\Omega}_{y}(\Stat_{y}(t))}}
        \nonumber\\
        &\Bracket{
            W^{\AP}_{ \tilde{\Omega}_{y}(\Stat_{y}(t)) }\Paren{\mathcal{R}(t+1)} +
            W^{\ES}_{ \tilde{\Omega}_{y}(\Stat_{y}(t)) }\Paren{\mathcal{Q}(t+1)}
        }.
        \label{eqn:partial}
    \end{align}
    }
    \label{problem_3}
\end{problem}

\begin{lemma}[]
    The optimal solution to Problem \ref{problem_2} and Problem \ref{problem_3} is same.
\end{lemma}

Specifically, the definitions of symbols in the remaining parts is given in the Appendix \ref{apped_1} and the calculation of Eqn (\ref{eqn:partial}) is elaborated below.
The expected value function $W^{\AP}_{\tilde{\Omega}_{y}(\Stat_{y}(t))}(\mathcal{R}(t+1))$ is easily obtained by evaluating the following equation
\begin{align}
    W^{\AP}_{\tilde{\Omega}_{y}(\Stat_{y}(t))}\Paren{\mathcal{R}(t+1)} &= \sum_{j\in\jSpace}\sum_{k\in\ccSet_{k}}\sum_{m\in\esSet_{k}}
    \nonumber\\
    &\Inorm{
        \Paren{ 1 - \gamma ({\Gamma}^{(k)}_{m,j})^{N} }^{-1} \hat{\vecG{\Theta}}^{(k)}_{m,j}(t+1)
    }.
    \label{w_ap}
\end{align}
% where $\hat{\Gamma}^{(k)}_{m,j} \define (\Gamma^{(k)}_{m,j})^{N}$.

The expression for expected value function is little more complex compared to Eqn (\ref{w_ap}).
It is affected with both arrival process under dispatching policy and last queue state.
However, we notice that the arrival process would be stationary after the maximum uploading time under the stationary baseline policy and the relationship between APs and edge server could be decoupled.
Hence, the expected value function $W^{\ES}_{\tilde{\Omega}_{y}(\Stat_{y}(t))}(\mathcal{Q}(t+1))$ is obtained by the following equation.
{\small
\begin{align}
    &W^{\ES}_{\tilde{\Omega}_{y}(\Stat_{y}(t))}\Paren{\mathcal{Q}(t+1)}
    = \sum_{j\in\jSpace}\sum_{m\in\esSet_{k}}\sum_{i=0,\dots,\frac{\Xi}{T}} \gamma^{i} \mathbb{E}[ Q_{m,j}({t+i+1}) ]
    \nonumber\\
    &~~~~~~~~~~~~~~~~~~~+ \gamma^{\frac{\Xi}{T}} \Paren{ \mat{I} - \gamma \hat{\mat{P}}_{m,j}(\tilde{\beta}_{m,j}) }^{-1} \vecG{\nu}({t+\frac{\Xi}{T}+1}) \vec{g}',
\end{align}   
}
where the $i$-th element of vector $\vec{g}$ denotes the cost of server as $Q_{m,j}(t)$, and
\begin{align}
    \hat{\mat{P}}_{m,j}(\tilde{\beta}_{m,j}) \define \prod_{n=0,\dots,N-1} \mat{P}_{m,j}(\tilde{\beta}_{m,j}),
\end{align}
% the probability distribution of $Q_{m,j}({t+i+1})$ is denoted by $\vecG{\nu}({t+i+1})$ which is obtained by calculating equation Enq. (\ref{eqn_0}) - Eqn (\ref{eqn_4}) ($\forall i=0,\dots,\frac{\Xi}{T}$);
where $\tilde{\beta}_{m,j}$ is the arrival distribution under baseline policy $\Pi(t)$ (on $m$-th ES with type $j$ jobs)
\begin{align}
    \tilde{\beta}_{m,j} &\define \sum_{k\in\apSet} \tilde{\lambda}^{(k)}_{m,j} \times \sum_{\xi=0,\dots,\Xi-1} \Pr\{ \xi<U_{k,m}\leq(\xi+1) \}
        \nonumber\\
    ~~~~&= \sum_{k\in\apSet} \tilde{\lambda}^{(k)}_{m,j}
\end{align}

The details of the procedure is elaborated as follows.
\begin{enumerate}
    \item Initialize the system dispatching policy $\Policy(\Stat(0))$ start with some heuristic policy;
    \item At the $1$-st broadcast time slot when $t=0$, the APs in set $\mathcal{Y}_{1}$ and the edge servers in the corresponding \emph{candidate server set} of APs in $\mathcal{Y}_{1}$, shall broadcast their LSI (including the heuristic policies);
    \item Each AP (say the $y_1$-th AP) in set $\mathcal{Y}_{1}$ would receive the OSI at the $D_{y_1}(1)$ time slots in the $1$-st broadcast interval, and then it updates its policy $\Omega_{y_1}(\Stat_{y_1}(1))$ by solving Eqn (\ref{eqn:partial}); The APs in set $\mathcal{Y}_{1}$ would keep with this policy until they receive the OSI again;
    \item At the $2$-nd broadcast time slot when $t=1$, the APs in set $\mathcal{Y}_{2}$ would receive the  broadcast OSI;
    % the $1$-st AP would broadcast $\Omega_{1}(\Stat_{1}(1))$ if it's in the \emph{conflict AP set} of the $2$-nd AP;
    \item Similarly, in the $t$-th broadcast interval the APs in subset indexed with $n \equiv (t + 1)\mod{N}$ would receive the OSI, and then update their policy $\Omega_{y_n}(t+1)$ ($\forall y_n\in\mathcal{Y}_{n}$) by solving Eqn (\ref{eqn:partial}).
\end{enumerate}

\begin{lemma}[Complexity Analysis]
    The computation complexity of solving Problem \ref{problem_3} is $O(J(c_k M_k+M_k))$ where $c_k, M_k$ is the number of APs in the \emph{conflict AP set} and number of edge servers in the \emph{potential server set} of the $k$-th AP, respectively.
    Overall, the computation complexity of the algorithm in one period is $O(J(KM+M))$.
\end{lemma}

\begin{lemma}[Performance Guarantee]
    % The optimized policy solved as $\Baseline(t)$ for next stage, is better than $\Baseline(t-1), \dots, \Baseline(1)$ when considering the \emph{approximated value function} defined above.
    The performance of the series of the baseline policies is upper bounded by the optimal solution $\Policy^*$ when considering the original Bellman's equation.
    $$
        V_{\Omega^*}\Paren{\Stat(t)}
        \leq W_{\tilde{\Policy}(\Stat(t))}\Paren{\Stat(t)}
        \leq W_{\tilde{\Policy}(\Stat(t-1))}\Paren{\Stat(t)},
        \forall \Stat(t)
    $$
\end{lemma}
\begin{proof}
    
\end{proof}

%----------------------------------------------------------------------------------------%
\delete{v11}{
    % [\IF, \ENDIF], [\FOR, \TO, \ENDFOR], [\WHILE, \ENDWHILE], \STATE, \AND, \TRUE
    % \begin{algorithm}[H]
    %     \caption{Online Iterative Policy Improvement Algorithm}
    %     \begin{algorithmic}[1]
    %         \STATE $t = 0$
    %         \FOR{$t = 1,2,\dots$}
    %             \STATE Evaluate $\Omega_0$ in \textbf{P1} according to Eqn (\ref{sp1})
    %             \FOR{$k \in \mathcal{K}$}
    %                 \STATE fix policy $\vec{\Omega}^{(k)}(t) \forall k' < k$
    %                 \STATE Evaluate $k$-th AP Local Policy $\tilde{\Omega}_k$ in \textbf{Pk} according to Eqn (\ref{sp2})
    %             \ENDFOR
    %         \ENDFOR
    %     \end{algorithmic}
    % \end{algorithm}
}
%----------------------------------------------------------------------------------------%