\section{System Model}
%----------------------------------------------------------------------------------------%
\subsection{Network Model}
We consider an edge computing system with $K$ Access Points (APs) and $M$ edge servers, which are connected in a network as illustrated in Fig.\ref{fig:system}.
The sets of APs and edge servers are denoted as $\apSet \define \set{1,\dots,K}$ and $\esSet \define \set{1,\dots,M}$, respectively.
Without loss of generality, it is assumed that there are $J$ types of computation jobs supported in this system, which are denoted via the set $\mathcal{J} \define \set{1,\dots,J}$.
Each AP collects the computation jobs from the mobile users within its coverage, and makes decision on the processing edge servers for each job type from the set $\esSet$.
It is assumed that the $k$-th AP only dispatches the computation jobs to the feasible edge servers, e.g. the edge servers within a certain number of hops.
Let $\esSet_{k} \subseteq \esSet$ be the set of edge servers, which can compute the jobs from the $k$-th AP and $\apSet_{m}$ be the set of APs, which may upload jobs to the $m$-th edge server.
We refer to $\esSet_{k}$ as the \emph{candidate server set} of the $k$-th AP, and $\apSet_{m}$ as the \emph{candidate AP set} of the $m$-th edge server ($\forall k\in\apSet, m\in\esSet$).
Different APs may have different candidate servers according to their locations in the network.
For example, it is possible that $\esSet_{k} \neq \esSet_{k'}$ and $\esSet_{k} \cap \esSet_{k'} \neq \emptyset$ when $k \neq k' \in \apSet$.
In this edge computing network, each AP decentralized makes job dispatching decision faciliated by information sharing from other APs and edge servers.
Specifically, each AP and edge server periodically broadcast their state information (e.g. edge server dispatching choice for each job type, list of jobs being uploaded from APs to edge servers, computing queue length and etc.), and one AP updates its strategy on job dispatching when receiving the broadcast state information.
In this paper, we shall optimize the job dispatching strategy at APs with part of the broadcast information and random broadcast latency.

\begin{figure*}[ht]
    \centering
    \includegraphics[width=0.80\textwidth]{system-model.pdf}
    \caption{The Illustration of MEC System Model}
    \label{fig:system}
\end{figure*}

%NOTE: [job space support and arrival process]
The time axis is organized by time slots.
The job arrivals in each time slot on each AP are modelled via independent Bernoulli distributions.
Specifically, the arrivals of the $j$-th job type at the $k$-th AP in different time slots are independent and identically distributed (i.i.d) Bernoulli random variables, and the arrival probability is denoted as $\lambda_{k,j}$ ($\forall k\in\apSet, j\in\jSpace$).
Let $A_{k,j}(t) \in \set{0,1}$ represents the event of job arrival, where $A_{k,j}(t)=1$ means one job of the $j$-th job type arrives on the $k$-th at the $t$-th time slot, and $A_{k,j}(t)=0$ means other wise.
Hence,
\begin{align}
    \Pr\{ A_{k,j}(t) = 1 \} = \lambda_{k,j}, \forall t,k\in\apSet,j\in\jSpace
\end{align}

%NOTE: [uploading process]
Each AP then immediately dispatches each type of received jobs to one edge server.
Different types of jobs may have different distributions on the input data size.
Moreover, due to the random traffic in the network, the job uploading from one AP to one edge server consumes a random number of time slots.
It is assumed that the distributions of uploading time are independent between any two uploading jobs.
Hence, we denote $\mathcal{U}_{k,m,j}$ as the uploading time for the $j$-th job type uploading from the $k$-th AP to the $m$-th edge server following some distribution with support $\set{1, \dots, \Xi}$, where $\Xi$ denotes the maximum uploading time ($\forall k\in\apSet, m\in\esSet, j\in\jSpace$).
In practice, the distribution of uploading latency may not be known to the APs or edge servers in advance.

%NOTE: [processing process]
There are $J$ virtual machines (VMs) running parallel on each edge server for the computation of $J$ job types, respectively.
For each job type, the uploaded jobs are computed in a First-Come-First-Serve (FCFS) manner.
Hence, a processing queue with maximum $L_{max}$ jobs is established for each VM.
The arrival jobs will be discarded when the processing queue is full.
Furthermore, we adopt the \emph{unrelated machines assumption} in \cite{tan-online} for job processing on edge servers.
Specifically, it is assumed that the computation time of different job types on different edge servers follows the memory-less Geometric distribution (with the unit of time slot) with different expectations.
Let $C_{m,j} \sim \mathbb{G}(1/c_{m,j})$ be the computation time distribution of the $j$-th job type on the $m$-th edge server, where $c_{m,j}$ is the expectation.
Hence, the probability mass function (PMF) of $C_{m,j}$ is given by:
\begin{align}
    f_{m,j}(k) \define (1-\frac{1}{c_{m,j}})^{k-1} \frac{1}{c_{m,j}}.
\end{align}
%----------------------------------------------------------------------------------------%

\subsection{Periodic Broadcast of State Information}
In order to facilitate cooperative dispatching for the APs, it is assumed that all the APs and edge servers will broadcast their \emph{local state information} (LSI) every $t_B$ time slots.
% as depicted in Fig.\ref{fig:brd-timeline}.
We shall refer to every $t_B$ time slots as a broadcast interval.
At the beginning of each broadcast interval (say the $t$-th broadcast interval), the LSI for AP and edge server is defined as follows, respectively.

%NOTE: State and Broadcast Information for AP
\begin{definition}[Local State Information of AP]
    One AP shall maintain LSI about the number of jobs in uploading, and the dispatching choice on edge servers for each job type.
    Specifically, at the $n$-th time slot in the $t$-th interval, the number of the $j$-th type job being uploaded $\xi$ time slots ago from the $k$-th AP to the $m$-th edge server is denoted as
    $R^{(k)}_{m,j,\xi}({t,n})$;
    at the $t$-th broadcast time slot, the dispatching choice of the $k$-th AP for proceesing of the $j$-th job type is denoted as
    $\omega_{k,j}(t) \in \esSet_{k}$ ($\forall k\in\apSet, m\in\esSet, j\in\jSpace, \xi\in(0,\Xi]$).
    
    Hence, the LSI of the $k$-th AP at the $t$-th broadcast is given as follows.
    \begin{align}
        \mathcal{R}_{k}(t) \define \set{\vec{R}^{(k)}_{m,j}(t,0), \set{\omega_{k,j}(t)|\forall j\in\jSpace} | \forall m\in\esSet_{k}, j\in\jSpace},
    \end{align}
    where
    $\vec{R}^{(k)}_{m,j}(t,0) \define ( R^{(k)}_{m,j,0}(t,0), \dots, R^{(k)}_{m,j,\Xi}(t,0) )$
    denotes the vector of random variables.
\end{definition}

%NOTE: State and Broadcast Information for ES
\begin{definition}[Local State Information of Edge Servers]
    One edge server shall maintain LSI about the computing queue status for each VM.
    Specifically, at the $n$ time slot in the $t$-th interval, the $m$-th edge server have $Q_{m,j}({t,n})$ denote the pending number of the $j$-th type job ($\forall m\in\esSet, j\in\jSpace$).

    Hence, the LSI of the $m$-th edge server at the $t$-th broadcast is defined as follows.
    \begin{align}
        \mathcal{Q}_{m}(t) \define \set{Q_{m,j}(t, 0) | \forall j\in\jSpace}.
    \end{align}
\end{definition}

We refer to \emph{global state information} (GSI) as the composition of all the broadcast information from all APs and edge servers in one broadcast and the definition is given as follows.
\begin{definition}[Global State Information]
    \begin{align}
        \Obsv^{\dagger}(t) \define
            \Brace{
                \mathcal{R}_{k}(t), \mathcal{Q}_{m}(t) | \forall k\in\apSet, m\in\esSet
            },
    \end{align}
    which is composed of all the broadcast information from all APs and edge servers at the $t$-th broadcast time slot.
\end{definition}

\begin{figure}[tp]
    \centering
    \includegraphics[width=0.45\textwidth]{images/conflict.pdf}
    \caption{The Example Illustration of Conflict Set and Partial Broadcast Information}
    \label{fig:conflict}
\end{figure}

%NOTE: Conflict of AP set and partial information definition
\comments{
    As the edge computing network resides in MAN, the propagation delay of LSI in such extensive network is not negligible.
    Thus for the $k$-th AP ($\forall k\in\apSet$), the reception latency of the LSI from the edge servers out of its \emph{candidate server set} $\esSet_{k}$ may be not acceptable.
    To alleviate the impact of information reception latency on decision making, we should leverage the definition of \emph{candidate server set} and \emph{candidate AP set}, and only use partial information from GSI for decision making for each AP.
}
Hence, we first define the \emph{conflict AP set} as follows.
\begin{definition}[Conflict AP Set]
    \begin{align}
        \ccSet_{k} \define \bigcup_{m\in\esSet_{k}} \apSet_{m}.
        % \ccSet_{k} \define \set{\forall k' \neq k\in\apSet|\esSet_{k'} \cap \esSet_{k} \neq \emptyset}
    \end{align}
    The \emph{conflict AP set} indicates that the subset of APs whose LSI could affect the decision making for the $k$-th AP.
\end{definition}

\begin{figure*}[t]
    \centering
    \includegraphics[width=0.80\textwidth]{brd-timeline.pdf}
    \caption{The timeline illustration of reception of OSI for the $1^{st}$ AP with the system setting in Fig.\ref{fig:conflict}.}
    \label{fig:brd-timeline}
\end{figure*}

It is assumed that each AP only collect the LSI from the APs in its \emph{conflict AP set} and edge servers from its \emph{candidate server set}.
For a simple example given in Fig.\ref{fig:conflict}, the $1^{st}$ AP only receives LSI from the $2^{nd}$ AP who is in its \emph{conflict AP set}, and LSI from the $1^{st}$ edge server who is in its \emph{candidate server set}.
Hence, we refer to the partial state information as \emph{observed state information} (OSI) and the definition is given as follows.
\begin{definition}[Observed State Information]
    The OSI for the $k$-th AP ($\forall k\in\apSet$) is defined as:
    \begin{align}
        \Obsv_{k} &= \set{\mathcal{R}_{k'} | \forall k'\in\ccSet_{k}}
                        \cup \set{\mathcal{Q}_{m} | \forall m\in\esSet_{k}}.
    \end{align}
\end{definition}

It is assumed that the $k$-th AP is able to collect its OSI $D_{k}(t)$ time slots later after the $t$-th broadcast time slot, where $D_{k}(t)$ is a random variable follows some distribution.
We refer to $D_{k}(t)$ as the \brlatency of the $k$-th AP for the $t$-th broadcast, which is upper bounded by the broadcast interval length $t_B$, i.e. $t_B > \hat{D}_{k}$ where $\hat{D}_{k}$ is the upper bound for $D_{k}(t)$ ($\forall k\in\apSet$).
Hence, $D_{k}(t)$ follows some distribution with support $\set{1,\dots,t_B}$, i.e. the broadcast interval $t_B$ is set that the $k$-th AP ($\forall k\in\apSet$) could receive the complete OSI before the start of next broadcast interval.

One AP shall update its dispatching decision immediately after the reception of its OSI.
An example is given to elaborate the decision as follows.
\begin{example}
    According to Fig.\ref{fig:brd-timeline}, we only consider the decisions update for the $1^{st}$ AP.
    In the first broadcast interval, it updates its dispatching decisions for all the job type $D_{1}(t)$ time slots later after the $t$-th broadcast, which is denoted as $\set{\omega_{1,j}(t+1) | \forall j\in\jSpace}$.
    At the start of the second broadcast interval, it will firstly keep the the previous decisions unchanged, and then updates the decisions immediately $D_{1}(t+1)$ time slots later which is denoted as $\set{\omega_{1,j}(t+2) | \forall j\in\jSpace}$.
    And then it repeats the process for the remaining of the time.
\end{example}

%----------------------------------------------------------------------------------------%

%----------------------------------------------------------------------------------------%