
\section{Introduction}

\delete{}{
    A number of emerging mobile applications, such as face recofnition, speech recognition and high-definition video rendering, are computation-intensive and delay-sensitive.
    Because of the limited computation resource and battery capacity, it is promising for the mobile devices to upload their computational0intensive jobs to edge servers with much more powerful computation capability.
    In this paper, we focus on the job dispatching optimization between multiple access points (APs) and edge servers via a network with random jobs uploading delay.
    Unlike cellular comunications, the delivery delay between APs and edge servers is hard to control due to unpredictable traffics (except for job uploading) in the network.
    We shall address the dispatching optimization in this scenario via a novel approximation Markov decision process (MDP) method, whose performance ca by analytically bounded.

    The scheduling algorithm design for edge computing systems has attracted tremendous research attentions.
    There have been a number of works considering the radio resource management for mobile edge computing systems.
    For example, the authors in \text{[1]} minimized the average energy consumption in a single-user system via Lyapunov optimization approach.
    In \text{[2]}, the authors derived the closed-form expressions of job uploading decisions and the allocation of computation and radio resources in a single-user system powered by wireless energy transfer.
    There are also a significant number of works considering the edge computing scenarios with multiple mobile users and single edge server.
        For example, the radio and computation resources allocation to guarantee user fairness and delay constraint in a multi-user system was considered in \text{[3]}.
        The authors in \text{[4]} proposed an optimal threshold-based uploading algorithm for mobile users.
        In order to minimize a weighted summation of total energy consumption and uploading delay, the authors in \text{[5]} proposed a distributed job uploading algorithm based on game theory.
        \hl{All these works considered the scheduling algorithm design in a wireless cell with single edge server.}
    Moreover, the authors considered joint optimization of service of caching and job uploading with multiple edge servers in \text{[6]}. Due to the limited storage space, edge servers can not process all job types.
    An online and decentralized scheduling algorithm is proposed to minimized the computation delay under total computation energy constraint.
    All the above works consider the scheduling of wireless transmission from mobile users to APs. In fact, there is also dispatching issue between APs and edge servers.
    For example, in a computer network, the job uploading delay from APs to edge servers is usually unpredictable, as it may be jammed by other traffics.

    There are also some works on the job dispatching design in computer networks.
    For example, without any job arrival information, the authors in \text{[7]} designed an online algorithm for job dispatching in edge computing systems to achieve a good competitive ratio.
    Given a predetermined network transmission delay, the edge server placement and static job dispatching are jointly optimized in \text{[8]} to minimize overall job uploading delay. The uploading delay and job computation time are predetermined in \text{[7,8]}.
    In practice, however, the network traffic between APs and edge servers is usually complicated.

}

Edge Computation is believed to be a promising technology for solving increasingly communication congestion in backbone network.
(Rich-media tasks are delay-sensitive).
\cite{Naha2018} is a survey about fog computing in delay-aware computing in IoT, and investigate numerous proposed computing architecture.
\fixit{(fully cooperative)}

Related works on job dispatching on scheduling in edge computing, mostly with centralized agent to apply action and seldom take delayed information impact into consideration.
There are some previous works on jobs scheduling strategies under the scenario of edge computation, together with joint optimization job dispatching and resource allocation:
\begin{itemize}
    \item \cite{Zheng2019} is a work considering maximizing the long-term utility in MEC offloading policy, and formulating the problem with MDP solved with Q-learning (However, it's applied with a centralized controller which would suffer from communication overhead, and without performance guarantee);
    \item \cite{Du2018} propose an offline algorithm with MINLP (Mixed-Integer Nonlinear Programming) problem formulation, considering min-max fairness guarantee in computation offloading and computation resource allocation in fog/edge computing scenario (However, with only one edge and one cloud node considered);
    \item \cite{Alameddine2019} is a work considering task offloading, scheduling and resource allocation joint optimization with Benders Decomposition (However, delay information is previous defined);
    \item \cite{Fan2017} considers cooperation of multiple MEC-BSs of computation offloading to other MEC-BSs (However, it doesn't consider the offloading utility impact on other MEC-BSs, i.e. only optimize for one BSs in the cluster).
\end{itemize}


Different from the previous referred works, which only consider optimization for one agent in system or using centralized agent for decision making, we focus on the impact of out-of-dated information on decision making in multiple agents distributed cooperation.
Due to frequent jobs releasing from users and uncertainty of jobs uploading time to servers, the obsolete information is inevitable in a purely distributed system.
This information sharing delay would cause severe mis-estimate for job dispatching decision making, and thus it will be hard to guarantee the cooperation among individual agents.
Moreover, the information sharing delay is often un-deterministic due to shared under-laid network topology, and offensively sharing strategy would cause network congestion to other normal functions.


As what we have learned for now, there are very limited discussions on this topic:
\begin{itemize}
    \item The earliest works entangling with out-of-dated information we could find is \cite{ref-01} (cited 167 times). In this work, the single agent is assumed not able to observe the global state, and thus they need communication to establish cooperation by sharing limited information. The agent considers communication as extra action to synchronize the states and thus incurs extra cost (However, the communication is without delay, thus converted into POMDP problem; criticize with impractical);
    \item The other work \cite{ref-02} considers continuous state observation with constant or stochastic delay with single agent;
    \item One researcher published a series of paper on this topic \cite{Lyu2017,Lyu2018,Lyu2018a,Lyu2018b}.
        \cite{Lyu2017} is work considering \emph{partial out-of-date knowledge} optimization in IoT computing scenario, with Lyapunov optimization;
        \text{[delay-sensitive, ToC]} \cite{Lyu2018} identify that task admission is critical to delay-sensitive applications in mobile edge computing, and proposes an (1-$\epsilon$)-approximation algorithm
        \text{[foggy, fully distributed online]} \cite{Lyu2018a} is a work fully distributed online optimization to minimize the time-average cost and achieve asymptotic optimality over infinite time;
        \cite{Lyu2018b} try to establish cooperation among selfish devices in fog computing, and out-of-date information is blamed for optimality gap but proved to asymptotically diminish with the proposed algorithm (However, the out-dated information is considered with limited effect);
\end{itemize}

%NOTE: [abandon useless references]
% Service Placement Scenario:
% \begin{itemize}
%     \item \cite{Rodrigues2017} is a work on minimizing service delay in mobile edge computing;
%     \item \cite{Yang2016} is a work considering services placement and requests dispatching on edge servers, and leverage users' pattern to predict "service cache" for online decision making;
%     \item \cite{Chen2018} is a work with SDN on task offloading and battery life saving, and solve the NLP problem with two sub-problems;
% \end{itemize}
% Using Game Theory:
% \begin{itemize}
%     \item \cite{yang2018} and \cite{Josilo2019a} considers distributed computation offloading game;
%     \item \cite{Liu2018} is a work considering minimize users' power consumption with Lyapunov optimization and matching theory;
%     \item \cite{Dinh2018} considers distributed multi-user offloading in wireless channel with selfish EPG (exact potential game);
%     \item \cite{Josilo2019} considers selfish offloading to achieve Nash equilibrium;
%     \item \cite{Chen2016} is a work considering multi-user computation offloading with multi-channel contention, and adopt game theory approach to achieve Nash equilibrium with upper bound of convergence time;
%     \item \cite{Zhang2018} considers multi-user offloading under transmit power decision and user association decision;
% \end{itemize}
% System Work:
% \begin{itemize}
%     \item \cite{Yu2018} is a system work published in ToMC, presents a framework to minimize remote execution overhead, and carry out real system experiments using large-scale data from cellular network provider;
%     \item \cite{Wang2018} is a system work published in IEEE Access, considers the mobility of mobile users in limited coverage solved with service migration and handover, and propose a framework;
% \end{itemize}

In this article, we consider there are Access Points (APs) in this network to connect the mobile users and edge servers.
The computation jobs would be released from UEs and the dispatching decisions are determined distributed on each AP nodes as depicted in Fig.\ref{fig:system}.
The information sharing scheme is proposed in this article with synchronized broadcast design. With the help of this scheme, we could immediately apply the job dispatching decision based on obsolete information, with some prior stochastic knowledge of global system.
Our contributions are summarized as follows.
\begin{itemize}
    \item To our best knowledge, we are the first work to propose the MDP framework making use of obsolete information in edge computing system. With shared prior stochastic knowledge about broadcast delay and uploading delay, the distributed agent could come to consensus on optimal policy adopting globally;
    \item We propose a global state MDP formulation to characterize the multiple-agent problem, i.e. single agent would consider multiple-agent decision optimality to achieve cooperation.
    We with adopt value function approximation to reduce the traditional algorithm complexity, and come up with performance guarantee performance bound.
\end{itemize}


The remainder of this paper is organized as follows.
In Sec.II, we justify the motivation on our topic.
And then we illustrate the system model and the optimization problem formulation in Sec.III and Sec.IV, respectively.
In Sec.V, we introduce a low-complexity MDP algorithm to the .
In Sec.VI, we show the results of evaluation.
In Sec.VII, the conclusion is given.