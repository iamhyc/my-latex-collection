
\section{Introduction}
%NOTE: General Background of MEC and Motivation
Edge computing is believed to be the solid solution for increasing computation-intensive and energy-hungry applications on mobile devices.
The edge server clusters are deployed in closer proximity to mobile devices than cloud infrastructure, which alleviates the communication overhead and enables {time-sensitive} and {deadline-aware} tasks offloading.
\comments{
    Edge Computation is believed to be a promising technology for solving increasingly communication congestion in backbone network.
    (large amount of mobile users wireless connected to the edge servers cluster, often in cooperation) \cite{MEC-SURVEY}.
    The edge servers are always with limited computation resources standalone, and efficient cooperation is indispensable among distributed servers.
}

%NOTE: Motivation with MAN
In this paper, we consider the edge computing network residing in Metropolitan Area Network (MAN), or called a city WAN \cite{MOBIHOC19-ZhouZ,tan-online,IOTJ18-FanQ}.
\comments{
    Due to frequent jobs releasing from users and uncertainty of jobs uploading time to servers, the obsolete information is inevitable in a purely distributed system.
    This information sharing delay would cause severe mis-estimate for job dispatching decision making, and thus it will be hard to guarantee the cooperation among individual agents.
    Moreover, the information sharing delay is often un-deterministic due to shared under-laid network topology, and offensively sharing strategy would cause network congestion to other normal functions.

    The new challenges exist that:
    1) the network traffic is random \needref{from-some-white-paper} while most existing literature only considers fixed environment \needref{the-fixed-environment};
    2) the centralized design is not possible in extensive network like MAN, while most existing literature considers the impractical centralized agent design without information exchange \needref{the-centralized-design};
    3) moreover, the staleness of information sharing is not negligible, and the information exchange over shared communication channel interfere with normal network traffic, so that: a) latency varies; b) broadcast storm \needref{elaboration-from-wiki}
}

%NOTE: Our contributions
In this article, we consider there are Access Points (APs) in this network to connect the mobile users and edge servers.
The edge computing network resided in Metropolitan area network (MAN).
The computation jobs would be released from UEs and the dispatching decisions are determined distributed on each AP nodes as depicted in Fig.\ref{fig:system}.
The information sharing scheme is proposed in this article with synchronized broadcast design. With the help of this scheme, we could immediately apply the job dispatching decision based on obsolete information, with some prior stochastic knowledge of global system.
Our contributions are summarized as follows.
\begin{itemize}
    \item To our best knowledge, this is the first work to propose a global standard MDP framework to characterize the stale information in edge computing network.
    With the broadcast information sharing scheme, the APs cooperatively dispatch the jobs;
    \item We propose a global state MDP formulation to characterize the multiple-agent problem, i.e. single agent would consider multiple-agent decision optimality to achieve cooperation.
    We with adopt value function approximation to reduce the traditional algorithm complexity, and come up with performance guarantee performance bound.
\end{itemize}

The remainder of this paper is organized as follows.
In Sec.II, we illustrate the system model and broadcast information sharing design.
In Sec.III, we formulate the global optimization problem of joint optimization on dispatching decisions of all APs.
In Sec.IV, we argue that the global joint optimization of all APs is impractical and then decouple the problem into optimization over individual APs.
We introduce a low-complexity algorithm in Sec.V and then show the numerical analysis in Sec.VI.
In Sec.VII, the conclusion is given.

\section{Related Works}
There are mainly three categories of problems in edge computing: resource allocation, job offloading, and service migration.
Specifically, the joint optimization problem is often addressed into consideration
%NOTE: resource allocation/placement (cache-like problem), service migration (impractical, \needref)

In this paper, we highlight the job dispatching problem which is NP-hard.

%NOTE: single-agent dispatching, single UE/server
Some work on job dispatching only with single user or single edge server available in the system.
% In \cite{TOC18-LyuX}, only one MEC is considered.

%NOTE: centralized multi-agent related works
Some work on job dispatching on scheduling in edge computing, mostly with centralized agent to apply action and seldom take delayed information impact into consideration.
\cite{ACCESS19-ZhengX} is a work considering maximizing the long-term utility in MEC offloading policy, and formulating the problem with MDP solved with Q-learning (However, it is applied with a centralized controller which would suffer from communication overhead, and without performance guarantee).
\cite{JSAC19-AlameddineHA} is a work considering task offloading, scheduling and resource allocation joint optimization with Benders Decomposition (However, delay information is previous defined).
\cite{Fan2017} considers cooperation of multiple MEC-BSs of computation offloading to other MEC-BSs (However, it doesn't consider the offloading utility impact on other MEC-BSs, i.e. only optimize for one BSs in the cluster).

%NOTE: decentralized multi-agent related works
Different from the previous referred works, which only consider optimization for one agent in system or using centralized agent for decision making, we focus on the impact of out-of-dated information on decision making in multiple agents distributed cooperation.
As what we have learned for now, there are very limited discussions on this topic.
% The earliest works entangling with out-of-dated information we could find is \cite{ref-01} (cited 167 times). In this work, the single agent is assumed not able to observe the global state, and thus they need communication to establish cooperation by sharing limited information. The agent considers communication as extra action to synchronize the states and thus incurs extra cost (However, the communication is without delay, thus converted into POMDP problem; criticize with impractical);
One researcher group has published a series of paper on this topic \cite{JSAC17-LyuX,TOC18-LyuX,TWC18-LyuX}.
In \cite{JSAC17-LyuX}, the authors firstly consider applying optimization in IoT scenario based on \emph{partial out-of-date knowledge}.
However, the definition of partial knowledge is actually partiall observed at one time, and the whole information will be learned at last.
Due to the Lyapunov optimization method adopted, the solution is actually not related to the information exchange process but converges to a stationary solution.
In \cite{TWC18-LyuX}, the authors consideres the information exchange scenario in distributed optimization scenario, but only selfish devices considered.
The out-of-date information is thus blamed for optimality gap but proved to be asymptotically diminish with the proposed algorithm.
However, the staleness of information gap could not be eliminated in complex fully-cooperative scenario, where all the agents share the same utility function to optimize.

%----------------------------------------------------------------------------------------%

\delete{v11}{
    \cite{Naha2018} is a survey about fog computing in delay-aware computing in IoT, and investigate numerous proposed computing architecture.

    Service Placement Scenario:
    \begin{itemize}
        \item \cite{Rodrigues2017} is a work on minimizing service delay in mobile edge computing;
        \item \cite{Yang2016} is a work considering services placement and requests dispatching on edge servers, and leverage users' pattern to predict "service cache" for online decision making;
        \item \cite{Chen2018} is a work with SDN on task offloading and battery life saving, and solve the NLP problem with two sub-problems;
    \end{itemize}
    Using Game Theory:
    \begin{itemize}
        \item \cite{yang2018} and \cite{Josilo2019a} considers distributed computation offloading game;
        \item \cite{Liu2018} is a work considering minimize users' power consumption with Lyapunov optimization and matching theory;
        \item \cite{Dinh2018} considers distributed multi-user offloading in wireless channel with selfish EPG (exact potential game);
        \item \cite{Josilo2019} considers selfish offloading to achieve Nash equilibrium;
        \item \cite{Chen2016} is a work considering multi-user computation offloading with multi-channel contention, and adopt game theory approach to achieve Nash equilibrium with upper bound of convergence time;
        \item \cite{Zhang2018} considers multi-user offloading under transmit power decision and user association decision;
    \end{itemize}
    System Work:
    \begin{itemize}
        \item \cite{Yu2018} is a system work published in ToMC, presents a framework to minimize remote execution overhead, and carry out real system experiments using large-scale data from cellular network provider;
        \item \cite{Wang2018} is a system work published in IEEE Access, considers the mobility of mobile users in limited coverage solved with service migration and handover, and propose a framework;
    \end{itemize}
}