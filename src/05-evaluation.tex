\section{Performance Evaluation}
\label{sec:evaluation}
In this section, we evaluate the performance of the proposed low-complexity dispatching policy $\tilde{\Policy}$ by numerical simulations.
The experiment setup and performance benchmarks are elaborated in Section \ref{subsec:setup}.
The simulation results are illustrated in Section \ref{subsec:basic}.
The sensitivity study on parameters is also applied to provide some insights on the robustness of the proposed policy in Section \ref{subsec:advance}.

\subsection{Experiment Setup}
\label{subsec:setup}
In the simulation, there are $K=5$ APs, $M=3$ edge servers and $J=5$ type of jobs in the system.
The edge servers are all fully-accessible to all the APs, i.e., $\esSet_{k}=\esSet$ ($\forall k\in\apSet$).

The time slot duration is 50 milliseconds and one broadcast interval consists of $t_{B}=20$ time slots, i.e., its duration is $1$ second.
% The distributions of arrival rate, uploading time and processing time are generated randomly.
The maximum uploading time is $3$ seconds, i.e., $\Xi = 3t_B$, and the distribution of $\mathbb{U}_{k,m,j}(\Xi)$ is arbitrarily generated within the support $[0, \Xi]$.
The expected computation time $c_{m,j}$ is also randomly generated in the range $[30,50]$ with the unit of time slot.

Each queue for VMs on edge server is with maximum queue length $L_{max}=20$, i.e., there would be at most $100$ jobs on one edge server.
The arrival rate is arbitrarily generated around $0.03$ for each AP.
The discount factor $\gamma$ is $0.95$ and the penalty weight $\beta$ is $20$.
% The arrival rate is taken as small probability with enough APs in the system, and correspondingly enough edge servers for the processing.

%-----------------------------------------------------------------------------------------------%
\begin{figure}[ht!]                                                                             %
    \centering                                                                                  %
    \includegraphics[width=0.45\textwidth]{41122-timeline-number-alter.pdf}                     %
    \caption{Illustration of number of jobs on all the APs and edge servers over time.}
    \label{fig:general_timeline}                                                                %
\end{figure}                                                                                    %
%-----------------------------------------------------------------------------------------------%

%NOTE: Benchmark Elaboration
We compare the proposed algorithm with other three heuristic algorithms which are listed as follows. For each type of jobs,
\begin{itemize}
    \item \textbf{Random Dispatching Policy}:
            Randomly choose a dispatching edge server in a time slot; 
    \item \textbf{Selfish Algorithm}:
            Always choose the edge server with the minimum expected uploading time together with the expected processing time;
    \item \textbf{Queue-aware Selfish Algorithm}:
            Always choose the edge server with the minimum expected uploading time together with the expected processing time and queueing time based on the observation of outdated queue states.
\end{itemize}
Specifically, we choose the \emph{Selfish Algorithm} which is state-invariant as the initial policy for our proposed algorithm.
The explicit definition is given as follows.
\begin{policy}[Selfish Policy]
    \begin{align}
        \Baseline \define \Brace{ \Pi_{k} |\forall k\in\apSet }, \Pi_{k} \define \Brace{{\pi_{k,j}|\forall j\in\jSpace}}
    \end{align}
    where $\pi_{k,j} \define \arg\min_{m\in\esSet_{k}} u_{k,m,j} + c_{m,j}$.
\end{policy}

%NOTE: Basic Performance
\subsection{Performance Analysis}
\label{subsec:basic}
As illustrated in Fig.\ref{fig:bar_plot}, the simulation results straightly show that the proposed 
algorithm outperforms other benchmarks all the time with less pending jobs in the system, and the system states converge fast from the start.
The detailed analysis is provided in Fig.\ref{fig:bar_plot} where three metrics are taken to demonstrate different profiles of the benchmarks.
The simulation results shown in Fig.\ref{fig:bar_plot} (a) 
The average JCT over all the jobs in Fig.\ref{fig:bar_plot}(b) resembles Fig.\ref{fig:bar_plot}(a) which proves the correctness of cost function selection.
\fixit{
    The metric as average cost in Fig.\ref{fig:bar_plot}(c),
}

\delete{v20}{
    The CDF of cost is illustrated in Fig.\ref{fig:cdf_cost} where the cost of \emph{SQF policy} is nearly the same as the \emph{selfish policy}, however, the \emph{SQF policy} actually have very poor job departure rate (, throughput) compared with the other's.
    This is because the broadcast information is periodically and outdated, and SQF could not handle the penalty brought by job rejection properly.
    And we take \emph{CDF of number of jobs} other than \emph{CDF of cost} which could better reflect the performance of the average JCT target in the simulation when job rejection considered.
}

%-----------------------------------------------------------------------%
\begin{figure}[ht]                                                      %
    \centering                                                          %
    \includegraphics[width=0.45\textwidth]{41122-bar-graph-alter.pdf}         %
    \caption{Illustration of performance metrics comparison with benchmarks.}
    \label{fig:bar_plot}                                                %
\end{figure}                                                            %
%-----------------------------------------------------------------------%
% \begin{figure}[ht]                                                      %
%     \centering                                                          %
%     \includegraphics[width=0.45\textwidth]{41122-cdf-cost.pdf}          %
%     \caption{Illustration of performance metrics comparison with benchmarks.}
%     \label{fig:cdf_cost}                                                %
% \end{figure}                                                            %
%-----------------------------------------------------------------------%
%----------------------------------------------------------------------------------------%
\subsection{Sensitivity Study}
\label{subsec:advance}  

%FIXME: replace the graphs
%-----------------------------------------------------------------------------------%
\begin{figure*}[ht!]                                                                %
    \centering                                                                      %
    \begin{tabular}{ccc}                                                            %
        \includegraphics[width=0.30\textwidth]{LowPressure-d0.pdf}&                 %
        \includegraphics[width=0.30\textwidth]{LowPressure-d1.pdf}&                 %
        \includegraphics[width=0.30\textwidth]{LowPressure-d2.pdf}                  %
        \\                                                                          %
        {\small (a) No \brlatency} &                                                %
        {\small (b) General \brlatency} &                                           %
        {\small (c) Maximum \brlatency}                                             %
    \end{tabular}                                                                   %
    \caption{Evaluation of Information Staleness Impact on Algorithm Robustness.}
    \label{fig:ss_delay}                                                            %
\end{figure*}                                                                       %
%-----------------------------------------------------------------------------------%


%-------------------------------------------------------------------%
\begin{figure}[hbt]                                                 %
    \centering                                                      %
    \includegraphics[width=0.45\textwidth]{Study-NumAPs.pdf}        %
    \caption{Illustration of impact of number of APs on algorithms.}
    \label{fig:ss_scale}                                            %
\end{figure}                                                        %
%-------------------------------------------------------------------%

%-------------------------------------------------------------------%
\begin{figure}[hbt]                                                 %
    \centering                                                      %
    \includegraphics[width=0.45\textwidth]{Study-ProcDist.pdf}      %
    \caption{Illustration of impact of uploading and processing time distribution on algorithms.}
    \label{fig:ss_dist}                                             %
\end{figure}                                                        %
%-------------------------------------------------------------------%

% %-------------------------------------------------------------------%
% \begin{figure}[hbt]                                                 %
%     \centering                                                      %
%     \includegraphics[width=0.45\textwidth]{bar_graph.pdf}           %
%     \caption{Illustration of impact of penalty factors on algorithms.}
%     \label{fig:ss_penalty}                                          %
% \end{figure}                                                        %
% %-------------------------------------------------------------------%

%NOTE: sensitivity study
\textbf{Various Signaling Latency.}
\fixit{
    The evaluation of staleness of \brlatency~is demonstrated in Fig.\ref{fig:ss_delay}.
}

\textbf{Number of APs.} %(a.k.a arrival rate)
The evaluation of scale of is demonstrated in Fig.\ref{fig:ss_scale}.
\fixit{
    It's shown that on the left of the figure, the SQF algorithm would work better when the system is almost idle; on the right of the figure, SQF and random algorithm could not handle high rejection rate and thus the \emph{average throughput} decreases extremely. 
}

\textbf{Uploading Time and Processing Time Distributions.}
\fixit{
    The evaluation of distribution of is demonstrated in Fig.\ref{fig:ss_dist}.
}

% \textbf{Penalty Factors.}
% \fixit{
%     The evaluation of CDF of number of dropped jobs (over the queue limit) is shown in Fig.\ref{fig:ss_penalty}.
%     % tends to job rejection or not
% }

%----------------------------------------------------------------------------------------%
