\begin{abstract}
    In this paper, the cooperative job dispatching problem in an edge computing network with multiple access points (APs) and edge servers is considered.
    \st{Specifically, the APs upload the received computation jobs to the edge servers according to the system state.}
    Due to the uncertain traffic of the network between APs and edge servers, the job uploading delay is random and unpredictable.
    Since each job dispatching decision will affect the system state in the future, we formulate the joint optimization of job dispatching at all the APs and all the scheduling time slots as an infinite horizon Markov decision process (MDP) problem.
    

    \fixit{
        Edge computing is believed to be the solid solution for increasing computation-intensive and energy-hungry applications on mobile devices.
        The edge server clusters are deployed in closer proximity to mobile devices than cloud infrastructure, which alleviates the communication overhead and enables \emph{time-sensitive} and \emph{deadline-aware} tasks offloading.
        The edge servers are always with limited computation resources standalone, and efficient cooperation is indispensable among distributed servers.
        However, due to intensive job arrival and unpredictable delay, the effect of stale information sharing would cause severe jobs misplacement and over-submission.

        In this work, we formulate the job dispatching problem in distributed Edge Computing system, and identify that the difficulty exists in how the decision is made decentralized based on stale information sharing.
        A periodic information sharing scheme via broadcast is proposed, and the standard MDP framework is introduced to depict the delay-aware dispatching process.
        The delicately designed low-complexity algorithm is adopted to provide an approximation solution against curse of dimensionality and with bounded analytical performance guarantee.
        The value function approximation and one-step policy iteration method is adopted to obtain a sub-optimal dispatching policy whose performance can be bounded analytically.
    }
\end{abstract}