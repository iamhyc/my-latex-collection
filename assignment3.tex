\documentclass{article}
\usepackage{mathrsfs}
\usepackage{amsmath}
\usepackage{amssymb}
\usepackage{braket}
\usepackage{xcolor}
\usepackage{enumitem}
\usepackage{float}
\usepackage[noend]{algorithm2e}
\usepackage[normalem]{ulem} % for strikeout line
\usepackage{graphicx}
\graphicspath{ {./images/} }
% \usepackage{epstopdf}

%-------------------------------------------------------%
\newcounter{pcounter}                                   %
\newenvironment{problem}                                %
{                                                       %
    \color{gray}                                        %
    \stepcounter{pcounter}                              %
    \textbf{\arabic{pcounter}.}                         %
}{}                                                     %
\newenvironment{solution}                               %
{\textbf{Solution:} }{$\blacksquare$}                   %
%-------------------------------------------------------%
\newcommand{\tab}{\ \ \ \ }                             %
\newcommand{\leadto}{\Rightarrow}                       %
\newcommand{\domR}{\mathcal{R}}                         %
\newcommand{\domS}{\mathbb{S}}                          %
\newcommand{\Gaussian}{\mathcal{N}}                     %
\newcommand{\IdenMat}{\textit{I}}                       %
\newcommand{\abss}[1]{\| #1 \|}                         %
\newcommand{\tr}[1]{\textbf{tr}(#1)}                    %
\newcommand{\vecOne}{\textbf{1}}                        %
\renewcommand{\vec}[1]{\mathbf{#1}}                     %
%-------------------------------------------------------%

\begin{document}
    %------------------- The Title -------------------%
    \parindent 0in
    \parskip 1em
    \title{COMP9501 Assignment 3 Solution Sheet}
    \author{HONG Yuncong, 3030058647}
    \maketitle

    \begin{section}{Problem 1}
        \setcounter{pcounter}{0}
        In a typical regression problem we have labels $\vec{y} = y_1, \dots, y_n$ of the function at points $\vec{X} = \{x_1, \dots, x_n\}$. Here $y_i=y(x_i) \in \domR, x_i = [x_i^{(1)}, \dots, x_i^{(d)}] \in \domR^{d}$. We are interested in estimating the value of the function $y_* = y(x_*)$, 
        \textbf{GPs} (Gaussian Process) provide a nice framework for obtaining an estimate for the \textbf{gradient and integral}, as well as quantifying the uncertainty.

        %=================== Problem 1.1 ===================%
        \begin{problem}
            [Gradients using Gaussian Processes] \\
            Recall the gradient at $x_*$ is a $d$-vector containing the $d$ partial derivatives. Denote the pratial derivatives of $y$ as $g_i(x) = \frac{\partial y(x)}{x^{(i)}}$, where $g_i : \domR^{d} \to \domR, i=1, \dots, d$. the gradient is $\vec{g}(x) = [g_1(x), \dots, g_d(x)]^T$. Write $g_{i*} = g_i(x_*)$ and $\vec{g}_* = g(x_*)$. \\
            We will model our observation as a Gaussian Process $y(x)$ over $\domR^d$ with zero mean and auto covariance $K$. We know that if a function $y$ is sampled from a GP, the distribution of $y_1, \dots, y_n, y_*$ at $x_1, \dots, x_n, x_*$ follow a Gaussian distribution. It can claso be shown that the gradient $\vec{g}_*$ and the function values $y_1, \dots, y_n$ are jointly Gaussian, i.e.
            $$
                [y, g_*] = [y_1, \dots, y_n, g_1(x_*), \dots, g_n(x_*)] \in \domR^{n+d}
            $$
            is also a Gaussian. This Gaussian has zero mean. The covariance $K_i$ between $y(x_1)$ and $g_i(x_2)$ and the covariance $K_{ij}$ between $g_i(x_1)$ and $g_i(x_2)$ are given respectively by
            \begin{gather*}
                K_i(x_1, x_2) = \frac{\partial K(x_1, x_2)}{\partial x_2^{(i)}}
                \\
                K_{ij} (x_1, x_2) = \frac{\partial^2 K(x_1, x_2)}
                                        {\partial{x_1^{(i)}} \partial{x_2^{(j)}}}
            \end{gather*}
            \begin{enumerate}[label=\alph*)]
                \item Let $\vec{K} \in \domR^{n \times n}$, $\vec{J} \in \domR^{n \times d}$, and $\vec{B} \in \domR^{d \times d}$ such that
                $$
                    \vec{K}_{ij} = K(x_i, x_j),
                    \vec{J}_{ij} = \frac{\partial K(x_i, x_*)}{\partial x_*^{(j)}},
                    \vec{B}_{ij} = \frac{\partial^2 K(x_*, x_*)}{\partial x_*^{(i)} \partial x_*^{(j)}}
                $$
                Write the (Gaussian) prior over $[\vec{y}, \vec{g}_*]$ in terms of $\vec{K, J, B}$ if we don't have any observation yet.

                \item Derive the posterior for $\vec{g_*}$ given that $\vec{y}$ was observed.
                
                \item Verify that the posterior mean of $\vec{g_*}$ is the same as the gradient of the posterior mean for $y(x_*)$.
            \end{enumerate}
        \end{problem}

        \begin{solution}
            \begin{enumerate}[label=\alph*)]
                \item % Answer for (a)
                As we know that $[\vec{y}, \vec{g_*}]$ follows Gaussian distribution with zero, then we could have:
                $$
                    y^* \triangleq [\vec{y}, \vec{g_*}] \sim \Gaussian(0, \Sigma))
                $$
                where the entries $\sigma_{i,j}$ of $\Sigma$ is covariance between $y^*_{i}$ and $y^*_{j}$; With respect to $\vec{K}, \vec{J}, \vec{B}$ we could represent $\Sigma$ in the following form:
                $$
                    \Sigma = \begin{pmatrix}
                        \vec{K}     &   \vec{J} \\
                        \vec{J}^T   &   \vec{B}
                    \end{pmatrix}
                $$
                Then we have the Gaussian prior over $[\vec{y}, \vec{g_*}]$ is:
                $$
                    Pr(\begin{pmatrix}
                        \vec{y}^T \\ \vec{g_*}^T
                    \end{pmatrix}) = \Gaussian(0, 
                        \begin{pmatrix}
                            \vec{K}     &   \vec{J} \\
                            \vec{J}^T   &   \vec{B}
                        \end{pmatrix}
                    )
                $$

                \item % Answer for (b)
                To derive the posterior distribution $p(\vec{g_*} | \vec{y})$, we leverage the properties learned in \textbf{Lecture 2} that:
                \begin{gather*}
                    p(\vec{x_1} | \vec{x_2}) = \Gaussian(\vec{x_1} | \mu_{1|2}, \Sigma_{1|2}) \\
                    \mu_{1|2} = \mu_1 + \Sigma_{12} \Sigma_{22}^{-1}(\vec{x_2} - \mu_2) \\
                    \Sigma_{1|2} = \Sigma_{11} - \Sigma_{12} \Sigma_{22}^{-1} \Sigma_{21}
                \end{gather*}
                As we know that $\vec{y}$ and $\vec{g_*}$ both follow Gaussian distribution with zero mean, then we could have:
                \begin{gather*}
                    \mu_{g_*|y} = \vec{J}^T \vec{K}^{-1} \vec{y} \\
                    \Sigma_{g_*|y} = \vec{B} - \vec{J}^T \vec{K}^{-1} \vec{J}
                \end{gather*}
                and $p(\vec{g_*} | \vec{y}) = \Gaussian(\vec{g_*} | \vec{J}^T \vec{K}^{-1} \vec{y}, \vec{B} - \vec{J}^T \vec{K}^{-1} \vec{J})$

                \item % Answer for (c)
                The posterior distribution for $[\vec{y(x_*)} | \vec{y(x)}]$ could be denoted by
                \begin{gather*}
                    p(\vec{y(x_*)}^T | \vec{y(x)}^T) = \Gaussian(\mu_{y_*|y}, \Sigma_{y_*|y})
                \end{gather*}
                \begin{align*}
                    \mu_{y_*|y} &= \Sigma_{y_* y} \Sigma_{y y}^{-1} \vec{y} \\
                    &= \Sigma_{y_* y} \vec{K}^{-1} \vec{y}
                \end{align*}
                Then we could have the gradient of this posterior as:
                \begin{align*}
                    \nabla \mu_{y_*|y} &= \frac{\partial \Sigma_{y_* y}}{\partial x_*} \vec{K}^{-1} \vec{y} \\
                    &= J^T \vec{K}^{-1} \vec{y}
                \end{align*}
                which is exactly the posterior mean of $g_*$
                
            \end{enumerate}
        \end{solution}

        %=================== Problem 1.2 ===================%
        \begin{problem}
            [Integration using Gaussian Process] \\
            Now we want to estimate the value of $\int_{a}^{b} y(t) dt$, i.e., the integral of a continuous 1-dimensional (i.e. $d=1$) function $y(x)$ over the interval $[a,b]$ from labels $\vec{y} = y_1, \dots, y_n$ at points $\vec{X} = \{x_1, \dots, x_n\}$ hwere $y_i = y(x)$ and $x_i \in \domR$. \\
            We define a function $Y(x) = \int_{0}^{x} y(t) dt$, and according to the Newton Leibniz formula, there is $\int_{a}^{b} y(t) dt = Y(b) - Y(a)$.
            \begin{enumerate}[label=\alph*)]
                \item We treat $Y$ as a GP with kernel $K_Y$, then the joint distribution of $y$ and $Y$ can be described by Gaussians. Using the properties described in Problem 1, outline a procedure to obtain this joint distribution $(\vec{y}, \vec{Y})$, where $\vec{Y} = [Y(b), Y(a)]^T$.\\
                Hint: you may want do define $K_y(x_1, x_2) = \frac{\partial^2 K_Y(x_1, x_2)}{\partial x_1 \partial x_2}$ and ${K_y}_{Y}(x_1, x_2) = \frac{\partial K_Y(x_1, x_2)}{\partial x_2}$.

                \item Derive the posterior $\vec{Y}|X, \vec{y}, a, b$ (which is also a Gaussian distribution);
                
                \item Derive the posterior distribution of the integral $\int_{a}^{b} y(t) dt$.
            \end{enumerate}
        \end{problem}

        \begin{solution}
            \begin{enumerate}[label=\alph*)]
                \item % Answer for (a)
                Firstly we define the covariance matrix $K_{yY}$ between $y$ and $Y$ as:
                \begin{gather*}
                    K_y(x_1, x_2) = \frac{\partial^2 K_Y(x_1,x_2)}{\partial x_1 \partial x_2} \\
                    K_{yY}(x_1, x_2) = \frac{\partial K_Y(x_1, x_2)}{\partial x_2}
                \end{gather*}
                As we know that the distribution of $(\vec{y}, \vec{Y})$ follows Gaussian distribuiton with zero means as:
                \begin{gather*}
                    \begin{pmatrix}
                        \vec{y} \\ \vec{Y}
                    \end{pmatrix}
                    \sim \Gaussian(0, \Sigma)
                    \\
                    \Sigma = \begin{pmatrix}
                        \Sigma_{yy} & \Sigma_{yY} \\
                        \Sigma_{Yy} & \Sigma_{YY} \\
                    \end{pmatrix}
                \end{gather*}
                where we denote each entry of covariance matrix with the definition above according to Problem 1 as:
                \begin{align*}
                    \Sigma_{yy} &= \vec{K} \\
                    \Sigma_{yY} &= \begin{pmatrix}
                        K_{yY}(x_1, b) & K_{yY}(x_1, a) \\
                        \dots & \dots \\
                        K_{yY}(x_n, b) & K_{yY}(x_n, a)
                    \end{pmatrix} \\
                    \Sigma_{Yy} &= \Sigma_{yY}^T \\
                    \Sigma_{YY} &= \begin{pmatrix}
                        K_Y(b,b) & K_Y(b,a) \\
                        K_Y(a,b) & K_Y(b,b)
                    \end{pmatrix}
                \end{align*}
                
                \item % Answer for (b)
                As for the posterior $Y|y, X, a, b$, we have
                \begin{gather*}
                    \mu_{Y|y} = \Sigma_{Yy} \Sigma_{yy}^{-1} \vec{y} \\
                    \Sigma_{Y|y} = \Sigma_{YY} - \Sigma_{Yy} \Sigma_{yy}^{-1} \Sigma_{yY}
                \end{gather*}
                and $Y|y, X, a, b \sim \Gaussian(\mu_{Y|y}, \Sigma_{Y|y})$
                
                \item % Answer for (c)
                According to the Newton Leivniz formula, we have
                $$
                    \vec{\hat{y}} \triangleq \int_{a}^{b} y(t) dt = Y(b) - Y(a)
                $$
                Then the posterior distribution of $\vec{\hat{y}}$ given observation of $\vec{y}$ following Gaussian distribution with:
                \begin{gather*}
                    \vec{\hat{y} | y} \sim \Gaussian(\hat{\mu_{Y|y}}, \hat{\Sigma_{Y|y}})
                    \\
                    \hat{\mu_{Y|y}} = (1,0)^T \mu_{Y|y} - (0,1)^T \mu_{Y|y} = (1,-1)^T \mu_{Y|y} \\
                    \hat{\Sigma_{Y|y}} = (1,-1)^T \Sigma_{Y|y} (1,-1)
                \end{gather*}
                
            \end{enumerate}
        \end{solution}
    \end{section}

    \begin{section}{Problem 2}
        \setcounter{pcounter}{0}
        By now you may be used to minimizing problems with respect to squared error loss. But there are many other loss functions. In this problem, we will investigate the regression problem using a different loss funciton.

        %=================== Problem 2.1 ===================%
        \begin{problem}
            [Quantile loss]\\
            \begin{enumerate}[label=(\alph*)]
                \item Let's define the following loss
                $$
                    p_{\tau}(z) = z \cdot (\tau - I(z < 0)) = 
                    \begin{cases}
                        z \cdot (\tau - 1) & \text{if $z < 0$} \\
                        z \cdot \tau & \text{if $z \geq 0$}
                    \end{cases}
                $$
                where $\tau \in (0,1)$ is called the $\tau$-th quantile, and $\IdenMat(z < 0)$ is the indicator function, i.e. $1$ if $z < 0$ and $0$ otherwise.
                \\
                Show that:
                $$
                    arg\ min_{\omega} \sum_{i} \rho_\tau (y_i - \omega) = y_{\tau}
                $$
                where $y_{\tau}$ is an observation sitting at the $\tau$-th top percentile of the observations (specifically, this means that $y_{\tau}$ is at leat exactly $\tau$ percent of the obervations).
                \\
                Hint: split the problem into positive and negative parts.

                \item When $\tau = 0.5$, this loss function has a well-known name in statistics. Whatis that?
            \end{enumerate}
        \end{problem}

        \begin{solution}
            \begin{enumerate}[label=(\alph*)]
                \item % Answer for (a)
                Firstly we denote the loss function as 
                \begin{align*}
                    l(\omega) &= \sum_{i} \rho_\tau (y_i - \omega) \\
                    &= (\tau - 1) \sum_{y_i < \omega} (y_i - \omega)
                        + \tau \sum_{y_i \geq \omega} (y_i - \omega)
                \end{align*}
                By setting the derivative of $l(\omega)$ to zero we have
                \begin{align*}
                    & \frac{d l(\omega)}{d \omega} = 0 \\
                    \Rightarrow & -(\tau-1)\sum_{i} I[y_i < \omega] + \tau \sum_{i} I[y_i \geq \omega] = 0 \\
                    \Rightarrow & \tau = \frac{\sum_{i} I[y_i < \omega]}{N}
                \end{align*}
                Then we have the optimal $\omega^* = y_\tau$, where $y_\tau$ is the $\tau$-th top percentile of the observations.

                \item % Answer for (b)
                When $\tau=0.5$ the loss function is in the following form:
                $$
                    \rho_{0.5}(z) = \begin{cases}
                        -0.5z & \text{if $z < 0$} \\
                        0.5z & \text{if $z \geq 0$}
                    \end{cases}
                $$
                which is the well-known $L_1$-loss function.
                
            \end{enumerate}
        \end{solution}

        %=================== Problem 2.2 ===================%
        \begin{problem}
            [Quantile regression]\\
            \begin{enumerate}[label=(\alph*)]
                \item Let $\{x_i\}_{i=1, \dots, N}$ be points in $\domR^K$ with outputs $\{y_i\}_{i=1, \dots, N}$. Let $X = \{x_1, \dots, x_N\}$. We define the regression quantile as 
                $$
                    \hat{\beta}(\tau) = arg\ \min_{\beta \in \domR^K} \sum_{i=1}^{N} \rho_\tau (y_i - x_i^T \beta)
                $$
                Prove that the solution of this problem is equivalent to the solution of the following linear program:
                \begin{gather*}
                    arg\ \min\limits_{\beta \in \domR^K, u,v \in \domR^N} u^T \vec{1} \tau + v^T \vec{1} (1- \tau) \\
                    \text{subject to } X^T \beta - y + u - v = 0 \\
                    u \geq 0 \\
                    v \geq 0
                \end{gather*}
                Hint: split the problem into positive and negative parts.

                \item Show that the dual of the above linear program is:
                \begin{gather*}
                    \max_z y^T z \\
                    \text{subject to } X z = (1 - \tau) X \vec{1} \\
                    z \in [0,1]^n
                \end{gather*}

                \item What does the value of $z_i$ in the dual problem tell us about $y_i - x_i^T \beta$ in the primal? specifically, using the KKT conditions, if $z_i=0$, what can you say about $y_i - x_i^T \beta$? What if $z=1$? What if $z \in (0,1)$
            \end{enumerate}
        \end{problem}

        \begin{solution}
            \begin{enumerate}[label=(\alph*)]
                \item % Answer for (a)
                Firstly we split the expression of $\hat{\beta(\tau)}$ in the two parts
                \begin{align*}
                    \hat{\beta}(\tau) &= \arg\min_{\beta \in \domR^K} \sum_{i=1}^{N} \rho_\tau (y_i - x_i^T \beta) \\
                    &= \arg\min_{\beta \in \domR^K} (\tau-1)\sum_{i=1}^{N} I[y_i < x_i^T \beta] (y_i - x_i^T \beta) + \tau \sum_{i=1}^{N} I[y_i \geq x_i^T \beta] (y_i - x_i^T \beta) \\
                    &= \arg\min_{\beta \in \domR^K} ([Y - \vec{X}^T \beta]_+ \vec{1}) \tau + ([\vec{X}^T \beta - Y]_+ \vec{1}) (1-\tau)
                \end{align*}
                Then we denote
                \begin{gather*}
                    u = [u_1, \dots, u_N], v = [v_1, \dots, v_N] \\
                    \text{where, } u_i = (y_i - x_i^T \beta) \\
                    v_i = (x_i^T \beta - y_i) \\
                    \text{s.t. } u - v = \vec{X}^T \beta - y \\
                    u \geq 0, v \geq 0
                \end{gather*}
                And then the original problem could written in the following Linear Program:
                \begin{gather*}
                    \arg\min\limits_{\beta \in \domR^K, u,v \in \domR^N} u^T \vec{1} \tau + v^T \vec{1} (1- \tau) \\
                    \text{subject to } X^T \beta - y + u - v = 0 \\
                    u \geq 0 \\
                    v \geq 0
                \end{gather*}

                \item % Answer for (b)
                Firstly we transform the above Linear Program into the canonical form:
                \begin{gather*}
                    \arg\min\limits_{\omega} c^T \omega \\
                    \text{s.t. } A \omega = b \\
                    u \geq 0 \\
                    v \geq 0
                \end{gather*}
                where, 
                \begin{gather*}
                    \omega = (\beta, u, v) \\
                    c = (0, \vec{1}\tau, \vec{1}(1-\tau)) \\
                    A = diag(X^T, -I_n, I_n)\\
                    b = y
                \end{gather*}
                Then we could directly have the dual problem of the linear program as
                \begin{gather*}
                    \arg\max_{z} y^T z \\
                    \text{s.t. } A^T z \leq c
                \end{gather*}
                From $A^T z \leq c$ we could have
                \begin{align*}
                    & A^T z \leq c \\
                    \Rightarrow& z \leq \vec{1}\tau, z \leq \vec{1}(1-\tau) \\
                                & X^T z = 0 \\
                    \Rightarrow& z \in [\vec{1}\tau - \vec{1}, \vec{1}\tau] \\
                                & X^T z = 0 \\
                    \Rightarrow& \tilde{z} \in [0, 1] \\
                                & X^T \tilde{z} = (1-\tau) \vec{X 1} \\
                        \text{where } &\tilde{z} = z + (1-\tau) \vec{1}
                \end{align*}
                And finnaly we write the daul problem in the following form:
                \begin{gather*}
                    \max_z y^T z \\
                    \text{subject to } X z = (1 - \tau) X \vec{1} \\
                    z \in [0,1]^n
                \end{gather*}

                \item % Answer for (c)
                With KKT condition,
                \begin{gather*}
                    \lambda^T [u, v] = 0 \\
                    \mu (x_i^T \beta - y_i + u_i - v_i) = 0\\
                    u, v, \lambda, \mu \geq 0
                \end{gather*}
                To conclude,
                $$
                    \begin{cases}
                        y_i - x_i^T \beta > 0, & z_i = 0 \\
                        y_i - x_i^T \beta < 0, & z_i = 1 \\
                        y_i - x_i^T \beta = 0, & z_i \in (0,1)
                    \end{cases}
                $$
                             
                if $z_i = 0$, $y_i - x_i^T \beta > 0$; \\
                if $z_i = 1$, $y_i - x_i^T \beta < 0$; \\
                if $z_i \in (0,1)$, $y_i - x_i^T \beta = 0$;
            \end{enumerate}
        \end{solution}
    \end{section}

\end{document}