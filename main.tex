\documentclass[10pt, conference, letterpaper]{IEEEtran}
\usepackage{cite}
\usepackage{xcolor,soul,framed}
\usepackage{amsmath,amsthm,amssymb,amsfonts}
\usepackage{algorithmic}
\usepackage{graphicx}
\usepackage{color, soul}
\usepackage{algorithm, algorithmic}
\usepackage[utf8]{inputenc}
\usepackage[english]{babel}
\usepackage{mathtools}
\graphicspath{ {./images/} }

%---------------------------------------------------------------%
\newtheorem{definition}{Denifition}
\newtheorem{assumption}{Assumption}
\newtheorem{problem}{Problem}
\newtheorem{lemma}{Lemma}
\newtheorem{corollary}{Corollary}
\newtheorem{example}{Example}
\newcommand{\domZ}{\mathbb{Z}_{*}}
\newcommand{\vecOne}{\mathbf{1}}
\newcommand{\ind}{\mathbf{I}}
\DeclarePairedDelimiter\set\{\}
%---------------------------------------------------------------%
\newcommand{\apSet}{\mathcal{K}}
\newcommand{\esSet}{\mathcal{M}}
\newcommand{\wSet}{\mathcal{W}}
\newcommand{\uSet}{\mathcal{U}}
\newcommand{\cSet}{\mathcal{C}}
\newcommand{\Stat}{\mathbf{S}}
%---------------------------------------------------------------%

\begin{document}

    %=============================== TITLE ===============================%
    \title{
        Meet-in-Future: Distributed Online Job Dispatching with Obsolete Information in Edge Computing System
    }
    \author{
        \IEEEauthorblockN{Yuncong Hong$^{*\dagger}$, Rui Wang$^{*}$, Haisheng Tan$^{\ddagger}$, Francis C.M. Lau$^{\dagger}$}
        \IEEEauthorblockA{
            $*$Southern University of Science and Technology, P.R. China,
            $\dagger$The University of Hong Kong, Hong Kong,\\
            $\ddagger$University of Science and Technology of China, P.R. China
        }
    }
    \maketitle

    %============================== ABSTRACT ==============================%
    \begin{abstract}
        \label{sec:abstract}
        Edge computing is believed to be the solid solution for time-sensitive big data real-time calculation. The cooperation among edge servers in the same coalition usually causes ineffective task scheduling due to obsolete information sharing, which is hard to tackle even with extra centralized agent design. In this work, we formulate the problem with job dispatching in distributed Edge Computing system, and identify the difficulty exists in cooperation between APs (Access Points) and ESs (Edge Servers) with delayed information. We design the broadcast information in the system and formulate the corresponding problem into a MDP problem. The value function approximation and \st{one-step policy iteration method is adopted to obtain a sub-optimal dispatching policy whose performance can be bounded analytically}.
    \end{abstract}

    % \begin{IEEEkeywords}
    %     Edge Computing, Job Dispatch, Delayed Information, Collective Observability, Distributed Multi-agent MDP
    % \end{IEEEkeywords}

    %============================ INTRODUCTION ============================%
    \begin{section}{INTRODUCTION}
        \label{sec:introduction}
        Our claims:
        \begin{itemize}
            \item Related works on scheduling in edge computing, mostly with centralized agent to apply action and seldomly take delayed information impact into consideration;
            \item Edge Server, Access Point, User Equipment; layered structure where decision is made distributedly on APs and computation is carried out on ESs; The AP-ES fully connected structure is reasonable, for example C-RAN to separate communication and computation 
            \item We identify the delayed system information is un-acceptable for explosion \emph{delay-sensitive jobs} in edge computing, and it's hard to establish cooperation among APs because of obsolete information;
            \item information sharing for cooperation is designed via (aligned) broadcast, job dispatch decision should be made immediately based on the previous collective information;
        \end{itemize}

        Our contributions:
        \begin{itemize}
            \item identify a instant job dispatch fully-distributed cooperative way, and avoid the centralized agent design;
            \item identify the uploading process may affect the heuristic greedy algorithm performance; identify the delay-information affected decision making;
            \item propose a global consensus state method to formulate the MDP problem (we use MDP definition in \cite{sutton1998introduction});
            \item adopt value function approximation to reduce the traditional algorithm complexity, and come up with distributed online learning algorithm;
        \end{itemize}

        Related works:
        \begin{itemize}
            \item The earliest related works we find is \cite{ref-01} (cited 167 times). In this work, the single agent is assumed not able to observe the global state, and thus they need communication to establish cooperation by sharing \emph{information}. The agent considers communication as extra action to synchronize the states and thus incurs extra cost. \\
            However, the communication is without delay, and converted into POMDP problem.
            \item The other work \cite{ref-02} considers continuous state observation with constant or stochastic delay with single agent. \\
            However, 
        \end{itemize}

    \end{section}

    %============================ SYSTEM MODEL ============================%
    \begin{section}{SYSTEM MODEL}
        \label{sec:model}
        \begin{subsection}{Network Model}
            The network topology of the MEC system considered is illustrated in Fig. \ref{fig:system}, which is composed of three elements. The user equipment (UE) is connected to one and only one access point (AP) all the time. The AP provides network access to all the UEs connected and accepts the released computing jobs from UEs; but AP itself is assumed to have no computation capability \st{because it could not carry various computation resources together with energy consumption with communication part (for example, C-RAN)}. So, AP would dispatch the job computation request to the backend edge servers (ES) in a distributed cooperative way.
            \begin{figure}[ht]
                \centering
                \includegraphics[width=0.45\textwidth, trim={0.5cm 0.5cm 0.5cm 0.5cm}, clip]{system-model.pdf}
                \caption{The Illustration of MEC System Model}
                \label{fig:system}
            \end{figure}

            Let $\mathcal{K} \triangleq \set{1,\dots,K}$ and $\mathcal{M} \triangleq \set{1,\dots,M}$ denote the set of APs and set of ESs in the MEC system respectively. We adopt \emph{timeslot} lasting for $\tau$ seconds as the minimum time slice in the system, and the dispatching and scheduling decision on AP and ES side respectively are all applied based on this timing. Furthermore, we assume the job arrival process for AP nodes is illustrated as follow:
            \begin{assumption}
                (Job Arrival Process at AP).
                We assume that the job arrival process $A_k(t)$ for $k$-th AP is i.i.d over the timeslots and each with Bernoulli distribution as $A_k(t) \sim Bernoulli(p_I)$.  According to \emph{Poisson Limit Theorem}, we identify the arrival process as memory-less between timeslots and with average arrival rate as $\mathbb{A_k(t)} = p_I$.
                This assumption implies that there will be at most one job arrives AP in one timeslot from all the UEs connected.
            \end{assumption}

            We identify \emph{delay} as the most unacceptable part in MEC system, so we take focus on uploading process and adopt well-designed broadcast to help APs come to consensus on global states.
            On the other side, a well-designed broadcast for all AP and ES nodes is required for information sharing. The broadcast is designed to be aligned at time start and with same broadcast interval as $T_{br}$. Due to the asymmetrical delay from one node to another, the information arrives at the decision node, $k$-th AP node with different delay as $x_{k,j}$
            \begin{align}
                x_{k,j} = T_{br} + d_{k,j}
            \end{align}
            where $j \in \apSet \cup \esSet$;
            \begin{itemize}
                \item for \emph{Parallel Uploading}: because the uploading time is finite and arrival frequency is at most 1 per timeslot, there actually needs finite channel capacity to serve the uploading process;
                uploading causes uncertainty;
                the total uploading process could be divided into Fig. \ref{fig:trans} between the adjacent timeslots, using some \emph{indicator functions};
                \item Information sharing delay is not acceptable, which will be elaborated in the following section \emph{Information Sharing Model};
            \end{itemize}

            The broadcast interval is globally synchronized and taken as integer timeslots;
            The aligned broadcast is guaranteed by timely synchronization protocol and is not the main component in our discussion.
            \begin{itemize}
                \item global aligned broadcast, and interval is far larger than the maximum delay from one nodes to the other AP nodes, i.e. $T \gg d$;
                \item Furthermore, consensus delay for $k$-th AP:
                    \begin{align}
                        \hat{d}_k = \max_{i\in(\mathcal{K} \cup \mathcal{N})}(d_{k,i})
                    \end{align}
            \end{itemize}
            And with the implication from the single broadcast, we generalize the conclusion for every broadcasts with:
            \begin{align}
                x_{k,j} = d_{k,j} + T^{br}_{k,j}
            \end{align}
            which takes a reasonable assumption that $T>d$ always ($j=k',n$ for $k$-th AP).
        \end{subsection}

        \begin{subsection}{Computation Model}
            \begin{itemize}
                \item unrelated machine model \cite{tan-online}, job length to characterize the resource preparation;
                \item deterministic rate and only one job at one time at servers' side;
                \item job length bounded by $L_C$; resource available is known because shared cooperation at initialization phase;
                \item job scheduling policy on all the server is identical, \emph{FCFS} \hl{(could be removed if we stress both dispatching and scheduling optimization)};
            \end{itemize}
            \begin{assumption}
                (Scheduling Policy)
            \end{assumption}
        \end{subsection}

        \begin{subsection}{Job Dispatch Model}
            The system features at one timeslot is composed of the jobs on all the APs and ESs in the MEC system. Specifically, we have waiting jobs set, uploading job set, and computing job set for the description.
            Let $\wSet_{k}(t) = \set{w_1, w_2, \dots, w_{N^w_k(t)}}, \forall k \in \apSet$ denotes the job set on $k$-th AP which comprises the jobs waiting for dispatch decision at $t$-th timeslot, where $w_i \in \domZ^M$ denotes the computing length on all $M$ unrelated edge servers. Then we have 
            \begin{itemize}
                \item \emph{Waiting Job Set} on $k$-th AP node in $t$-th timeslot as: ; .
                \item \emph{Uploading Job Set} from $k$-th AP to $m$-th ES in $t$-th timeslot as: $\uSet_{k,m}(t) = \set{u_1, u_2, \dots, u_{N^u_{k,m}(t)}}, \forall k \in \apSet, m \in \esSet$; where $u_i \in (L_C), L_{cd}^{(U)}(t)$.
                \item \emph{Computing Job Set} on $m$-th ES in $t$-th timeslot as: $\cSet_{m}(t) \triangleq \set{L_{cd}^{(C)}(t)}$;
            \end{itemize}

            While the jobs processing is carried out following a descriptive procedure in the model section, we need some denotations to characterize the intrinsic of the transition. Firstly We come up with denotations for transition job sets expressed in figure \ref{fig:trans} as:
            \begin{align}
                I^{(W \to W')}_{k}(t) & \triangleq \{ j | a^{(k)}(j)=0, \forall j \in S^{'(W)}_k(t)\}
                \\
                I^{(W \to U)}_{k,n}(t) &\triangleq \{ j | a^{(k)}(j)=n, \forall j \in S^{'(W)}_k(t)\}
                \\
                I^{(U \to C)}_{k,n}(t) &\triangleq \{j|(L^{(U)}_{cd})_j=1, \forall j \in S^{(U)}_{k,n}(t)\}
                \\
                I^{(U \to U')}_{k,n}(t) &\triangleq S^{(U)}_{k,n}(t) \backslash I^{(U \to C)}_{k,n}(t)
                \\
                I^{(C \to \Phi)}_{n}(t) &\triangleq \{j|\arg\min_{j} (L^{(C)}_{cd})_j \wedge (L^{(C)}_{cd})_j=1\}
                \\
                I^{(C \to C')}_{n}(t) &\triangleq S^{(C)}_{n}(t) \backslash I^{(C \to \Phi)}_{n}(t)
            \end{align}
            \begin{figure}[h]
                \centering
                \includegraphics[width=0.45\textwidth]{single-transition.png}
                \caption{Single-step transition function composing graph}
                \label{fig:trans}
            \end{figure}
            Then we establish the relationship between those indicator functions with $S(t+1)$:
            \begin{align}
                \begin{cases}
                    S^{(W)}_{k}(t+1) = I^{(W \to W')}_{k}(t)
                    \\
                    S^{(U)}_{k,n}(t+1) = I^{(W \to U)}_{k,n}(t) + I^{(U \to U')}_{k,n}(t)
                    \\
                    S^{(C)}_{n}(t+1) = \sum_k I^{(U \to C)}_{k,n}(t) + I^{(C \to C')}_{n}(t)
                \end{cases}
            \end{align}
        \end{subsection}
    \end{section}

    %============================ FORMULATION =============================%
    \begin{section}{FORMULATION}
        \label{sec:formulation}
        \begin{section}{System State and Scheduling Policy}
            We formulate the standard MDP problem in this section; a fully distributed way.

            \begin{definition}
                (System State).
                $T_t = (\Stat_t, \Stat_{t-1}, C_t)$, $t=1,2,\dots$; where the transition for $(S_{t}, S_{t-1})$ is with policy $\vec{\Omega}(\tilde{S}_i)$ elaborated in the last section.
            \end{definition}

            \begin{definition}
                (Scheduling Policy).
                Same as the definition in last section.
                \begin{align}
                    \vec{\Omega}(T_t) = \vec{\Omega}(\tilde{S}_t) \equiv \vec{\Omega}_{\Delta{t}}(S_t, S_{t-1})
                \end{align}
            \end{definition}
        \end{section}

        \begin{subsection}{The Optimization Problem}
            In our problem, we consider \emph{response time} of jobs;
            While $C_t$ denotes the cost collected in the last broadcast interval, we have the expression as:
            \begin{align}
                C_t = \sum_{t'\in[1,T^{br}]} \sum_{s'} P_{S_{t-1},s'}^{t'}(\vec{\beta}_{\Delta{t}}) \times |s'|
            \end{align}
            where,
            \begin{align}
                P_{S_{t-1},s'}^{t'}(\vec{\beta}_{\Delta{t}})
                = \{
                    \prod_{\Delta{t} \in [1,t']} P_{ss'}(\vec{\beta}_{\Delta{t}})
                \}_{S_{t-1}, s'}
            \end{align}
            
            \begin{problem}
                (Distributed Cooperative Job Dispatching)
                \begin{gather}
                    \min_{\Omega} \lim_{T \to \infty} E[\frac{1}{T} \sum_{t=0}^{T} N(t)]
                    \nonumber\\
                    N(t) = \sum_{k \in K} (A^{(k)}(t) + N_k(t))
                            + \sum_{n \in N} N_n(t)
                \end{gather}
            \end{problem}
            where $N_k(t)$ denotes the number of jobs on $k$-th AP, and $N_n(t)$ denotes the number of jobs on $n$-th ES.
            The goal to minimize the cost caused by ESs could be simply achieved with heuristic algorithm \emph{SJF} (shortest-job first), because it is the fastest way to reduce number of jobs on the edge server. So, in the next MDP problems we will focus on the policy applied on AP side, and leave ES with fixed heuristic algorithm.

            The \emph{Bellman Equation}:
            \begin{align}
                V(T_{t}) = C_t + \gamma \sum_{T_{t+1}} \Pr\{T_{t+1}|T_{t}, \vec{\Omega}(T_t)\} \cdot V(T_{t+1})
            \end{align}
        \end{subsection}

        \begin{subsection}{The Transition Elaboration}
            \begin{definition}
                (Single Step Transition).

            \end{definition}

            \begin{definition}
                (Multi Step Transition).
            \end{definition}
        \end{subsection}
        
    \end{section}

    %============================= ALGORITHM ==============================%
    \begin{section}{LOW-COMPLEXITY SOLUTION}
        \label{sec:algorithm}
        As the formulated problem above is of infinite states and the action space would be exponentially expanded with respect to number APs and ESs, we could not use traditional \emph{policy iteration} or \emph{value iteration} algorithm \cite{sutton1998introduction} for unacceptable computational complexity. To alleviate curse of dimensionality, we take one baseline policy to approximate the value function as $\tilde{V}(T_t)$ and carry out one-step iteration to come up with a better value function approximation.
        \hl{Traditional value iteration is intractable due to the following reasons: (1) the number of active devices is not fixed and the state space grows exponentially with the increasing number of active devices; (2) the spaces of small-scale fading and path-loss are continuous.}

        \begin{subsection}{Baseline Policy}
            \begin{problem}
                (Fixed FCFS Optimization).
            \end{problem}

            \begin{definition}
                (Job Arrival Process at $n$-th ES).
            \end{definition}

            At last, we collect the cost according to the definition in our MDP problem but with respect to our approximate algorithm.
            \begin{align}
                & \tilde{V}^{\pi}(T_t)
                \nonumber%\\
                = E_{\pi} \{ \tilde{C}_{t} + \gamma \tilde{C}_{t+1} + \gamma^2 \tilde{C}_{t+2} + \dots |S_{t-1}=s \}
                % \nonumber\\
                % = & \sum_{k=0}^{\infty} \gamma^{k} \sum_{t'=kT^{br}+1}^{(k+1)T^{br}} \sum_{s'} \tilde{P}^{t'}_{s,s'} \times |s'|
            \end{align}
            % where $\tilde{P}_{s,s'}$ is fixed under the given policy $\vec{\beta}_{\pi}$.
        \end{subsection}

        \begin{subsection}{The Distributed Algorithm}
            Then we introduce the one-step iteration algorithm in this section:
            % [\IF, \ENDIF], [\FOR, \TO, \ENDFOR], [\WHILE, \ENDWHILE], \STATE, \AND, \TRUE
            \begin{algorithm}[H]
                \caption{Distributed Algorithm for $k$-th AP}
                \begin{algorithmic}
                    \WHILE{\TRUE}
                        \STATE (in progress)
                        % \FOR{$k \in \mathcal{K}$}
                        %     \STATE fix policy $\vec{\Omega}^{(k)}(t) \forall k' \neq k$
                        % \ENDFOR
                    \ENDWHILE
                \end{algorithmic}
            \end{algorithm}
        \end{subsection}
        
    \end{section}

    %============================ EVALUATION ==============================%
    \begin{section}{EVALUATION}
        \label{sec:evaluation}
        (in progress)
    \end{section}

    %============================= CONCLUSION =============================%
    \begin{section}{CONCLUTION}
        \label{sec:conclusion}
        The future work to mention:
        \begin{itemize}
            \item non-aligned broadcast
            \item broadcast failure
            \item randomized broadcast delay
        \end{itemize}
    \end{section}

    %============================== REFERENCE =============================%
    \bibliographystyle{IEEEtran}
    \bibliography{main.bib}
\end{document}